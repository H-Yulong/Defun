% !TEX root = ../main.tex

Dependent types are used to verify the correctness of large-scale programs, as they can express program specifications with their type system and guarantee that all specifications hold by type-checking.
A modern trend in compiling dependently-typed languages is to turn source programs into target programs in a dependently-typed intermediate language and preserve the type information~\cite{bowman2019compiling}. In this way, program specifications are preserved through types and checking the correctness of separately compiled and linked programs becomes possible. Type-preserving compilation is well-studied in non-dependently-typed scenarios, and the current challenge is adapting compiler transformations for non-dependent types to dependent types.
	
This dissertation studies \textit{defunctionalization}, a program transformation that turns a higher-order functional program into a first-order one by replacing every function with a \textit{label} (a special kind of first-class value). Type-preserving defunctionalization for simply-typed and polymorphic systems is well-established in the literature~\cite{DBLP:conf/icfp/BellBH97, DBLP:conf/popl/PottierG04}, and my work extends defunctionalization further to dependently-typed systems.

This introduction motivates the development of type-preserving compilers for dependently-typed languages. It then summarizes the contributions of this dissertation and discusses related works in type-preserving compilation.

\section{Motivation and contributions}

Dependently-typed systems are used for verifying the functional correctness of a range of large-scale software, from the CompCert C compiler~\cite{DBLP:conf/popl/Leroy06} to the CertiKOS OS kernel~\cite{DBLP:conf/popl/GuKRSWWZG15}. 
In dependently-typed systems, types may contain term variables and the type of a function’s output could depend on the \textit{value} of its input. The type of a \textit{dependent function} is denoted as $\pitype{x}{A}{B}$, where $B$ is called an \textit{indexed family} of types (indexed by terms of type $A$).
With the \textit{proposition-as-types} principle, a dependently-typed calculus is both a programming language and a logic. So, it is possible to specify invariants and Hoare-logic style pre- and post-conditions for programs as types, and dependent type-checking guarantees that the specified conditions hold. For example, I can specify the following type for a safe natural-number division function in Agda, which takes arguments \texttt{x}, \texttt{y}, and an proof that \texttt{y} is non-zero to type-check. 
\begin{lstlisting}
	(* Returns x / y *)
	div : (x : Nat) -> (y : Nat) -> (p : y > 0) -> Nat
\end{lstlisting}

However, correctness is not guaranteed for separately compiled programs. For example, programmers may break specifications by linking incompatible pieces of separately compiled code. There is no way to check the specifications of separately compiled and linked programs, since the current implementation of compilers for dependently-typed languages removes type information after compiling. In other words, there is no way to enforce correctness on the target code with dependent types and type-checking.

This problem can be solved with type-preserving compilation, a well-known technique in non-dependently-typed scenarios.
These compilers preserve function behaviours and program semantics like regular compilers. Furthermore, they preserve the type information – well-typed source programs are turned into well-typed target programs in a typed intermediate language. It is possible to type-check the target code and ensure correctness statically in separately compiled and linked programs using the preserved types. Type information is only removed after the program is fully linked and checked. The current challenge of developing type-preserving compilers for dependently-typed languages is adapting non-dependently-typed compiler transformations to dependent types. 

Defunctionalization is a compilation technique that turns higher-order functional programs into first-order ones. It is well-studied in simply-typed and polymorphic scenarios, and my work extends defunctionalization further to dependently-typed systems.

\paragraph{Contributions} 

The main contribution of this dissertation is \textit{abstract defunctionalization}, a method of performing type-preserving defunctionalization with dependent types.
Abstract transformation consists of a target language with a primitive notion of function \textit{labels} that fits the abstract description of defunctionalization, and a transformation from the source language into the target language. 
I prove that the transformation is type-preserving and correct, and the target language is type-safe and consistent.

The remainder of this dissertation is structured as follows. In Chapter 2, I explain the standard method of defunctionalization for simply-typed and polymorphic programs and defines the source language of dependently-typed defunctionalization. In Chapter 3, I illustrate that Pottier and Gauthier's standard technique of polymorphic defunctionalization~\cite{DBLP:conf/popl/PottierG04} does not extend to dependently-typed systems. 

In Chapter 4, I present abstract defunctionalization as an alternative method. This chapter formally presents the Defunctionalized Calculus of Constructions (DCC), the target language of abstract defunctionalization. It also defines the defunctionalization transformation from the source language into DCC. Chapter 5 presents the proof of type preservation and correctness of the transformation and shows that DCC is consistent and type-safe. Chapter 6 concludes the dissertation with a closing remark on implementing DCC in OCaml and future work.

\section{Related work in type-preserving compilation}

Type-preserving compilation was initially developed for optimizing and verifying the compiled code, and it is a well-known technique in compiler design for languages that are not dependently-typed. Tarditi et al.~\cite{DBLP:conf/pldi/TarditiMCSHL96} developed TIL (typed intermediate language), an ML compiler featuring type-directed code optimization of loops, garbage collections, and polymorphic function calls. 

Morrisett et al.~\cite{DBLP:journals/toplas/MorrisettWCG99} developed a type-preserving translation from System F to TAL (typed assembly language). Later, Xi and Harper~\cite{DBLP:conf/icfp/XiH01} introduced DTAL, which extends TAL with a restricted form of dependent types, making it suitable for compiling a variant of ML with dependent datatype extensions. Necula’s proof-carrying code~\cite{DBLP:conf/popl/Necula97} is another early method for generating reliable executables. It relies on an external logical framework to check the correctness of proofs attached with the code.

Recently, Bowman et al. contributed a series of work about compiling with dependent types, including the type-preserving continuation-passing style (CPS) and closure conversion transformations~\cite{DBLP:journals/pacmpl/BowmanCRA18,DBLP:conf/pldi/BowmanA18}.



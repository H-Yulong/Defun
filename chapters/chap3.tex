% !TEX root = ../main.tex

This chapter shows that the polymorphic defunctionalization technique cannot be adapted to dependent types. Section~\ref{sec:3.1} introduces \textit{inductive families}, a more general version of GADTs in dependently-typed languages. Section~\ref{sec:3.2} proves that it is impossible to define type-preserving defunctionalization using inductive families.

\section{Inductive families}
\label{sec:3.1}

This section introduces inductively defined data types (inductive families), and I present code examples in Agda, a dependently-typed programming language with a Haskell-like syntax. No knowledge of Agda is required to understand the contents of this section, except for two things:
\begin{itemize}
  \item In Agda, there is an infinite hierarchy of predicative universes \[ \mathtt{Set_0 : Set_1 : Set_2 : \cdots} \] similar to the hierarchy in CC.
  \item The dependent function type $\pitype{x}{A}{B}$ is written as \texttt{(x:A) -> B} in Agda's syntax.
\end{itemize}

Agda supports inductive families, which are essentially inductively defined data types indexed by terms~\cite{AgdaDoc}. Their constructors may return elements in an arbitrary type of the indexed family. Example~\ref{ex:fin} shows the definition of \texttt{Fin}, the inductive family of finite sets.

\begin{exmp} \texttt{Fin} is a data type indexed by natural numbers (\texttt{Nat}) such that there are exactly \texttt{n} valid expressions of type \texttt{Fin n}. Given a natural number \texttt{n}, \texttt{fzero}constructs an element of type \texttt{Fin n+1}; given a natural number \texttt{n} and an element of type \texttt{Fin n}, \texttt{fsucc} constructs an element of type \texttt{Fin n+1}.
\begin{lstlisting}[escapechar=@]
data Fin : Nat -> @\codemath{Set_0}@ where
    fzero : (n : Nat) -> Fin (n + 1)
    fsuc : (n : Nat) -> Fin n -> Fin (n + 1)
\end{lstlisting}
\label{ex:fin}
\end{exmp}
GADTs can be considered as a limited version of inductive families that can only be indexed by types. In general, inductive families are defined in the following form.
\begin{lstlisting}[escapechar=@]
data D : (@\codemath{y_1}@ : @\codemath{T_1}@) -> @\codedots@ -> (@\codemath{y_n}@ : @\codemath{T_n}@) -> @\codemath{Set_d}@ where
    @\codemath{c_1} : \codemath{A_1}@
    @\codemath{c_2} : \codemath{A_2}@
    @\codedots@ 
\end{lstlisting}

\texttt{D} is an inductive family in \codemath{Set_d} indexed by terms of type \codemath{T_1, \cdots, T_n}. For each constructor \codemath{c_i}, it takes a number of arguments \texttt{(\codemath{z_i:S_i})} and constructs an element of type \texttt{(\codemath{D\ t_1\ \cdots \ t_n})}, where each \codemath{t_i} is an expression of type \codemath{T_i}. Concretely, each \codemath{A_i} takes the form of
\begin{lstlisting}[escapechar=@]
(@\codemath{z_1}@ : @\codemath{S_1}@) -> @\codedots@ -> (@\codemath{z_m}@ : @\codemath{S_m}@) -> (D @\codemath{t_1}@ @\codedots@ @\codemath{t_n}@).
\end{lstlisting}
Note that arguments to \texttt{D} and \texttt{c} are dependently typed -- \codemath{T_{i+1}} could mention \codemath{y_1 \cdots y_i}, and \codemath{S_{i+1}} could mention \codemath{z_1 \cdots z_i}. Also, \codemath{y_1 \cdots y_n} are not bound in \codemath{A_i}.

Not all inductive definitions are consistent -- some contain impredicative constructors (the constructor's universe is larger than the data type's), and some admit non-terminating functions. These definitions are sources of paradoxes, which Agda excludes by performing a \textit{universe check} and a \textit{positivity check} on inductive definitions. Take the data type \texttt{D} defined above as an example, it must satisfy two additional conditions:
\begin{itemize}
  \item Types of its constructors must fit in the universe of \texttt{D}.
  \item \texttt{D} must occur \textit{strictly positively} in types of its constructor's arguments.
\end{itemize}

Concretely, the universe of every argument type \codemath{S_i} (the type of \codemath{S_i}) should be smaller than \codemath{Set_d} to pass the universe check. Having strict positivity means that constructors cannot return self-indexing types like \texttt{(D (D \codedots )} \texttt{\codedots )}, and every argument type \codemath{S_i} must either:
\begin{itemize}
  \item not mention \texttt{D} at all.
  \item mention \texttt{D} in the form of \texttt{\codemath{S_i} = (\codemath{x_1 : C_1}) -> \codemath{\cdots} -> (\codemath{x_k : C_k}) -> D}, where \texttt{D} does not occur in any \codemath{C_j} (arguments to \texttt{D} omitted).
\end{itemize}

The requirements for strict posivitity and universes are shared across many dependently-typed languages that support inductive families, like Coq's Gallina~\cite{?} or Timany and Sozeau's pCuIC~\cite{?}. 

\section{Failure to defunctionalize}
\label{sec:3.2}

In this section, I illustrate that it is impossible to adapt the method of performing polymorphic defunctionalization to dependently-typed languagesusing inductive families. Chapter 2 illustrated that type-preserving defunctionalization can be done with ADTs for simple types and GADTs for polymorphic types. In polymorphic defunctionalization, a function of type \texttt{a -> b} is translated to a constructor of a GADT \texttt{lam}, and the constructor's return type is \texttt{(a, b) lam} (Example 2.1.2). Similarly, we can try to define an inductive family \texttt{Pi}, and translate each dependent function of type $\pitype{x}{A}{B}$ to a constructor of \texttt{Pi A B} (suppose that \texttt{Pi} has a way of encoding \texttt{B}'s dependency on terms of type \texttt{A}). 

Suppose the source language is CC and the target language is Agda. Let $\bbracket{-}$ denote the defunctionalization transformation. I assume that $\bbracket{\lam{x}{A}{e}} =$ \texttt{Fun}, i.e.~$\bbracket{-}$ turns a function in source to a unique constructor \texttt{Fun} of an inductively defined data type in target, and each free variable in $\lam{x}{A}{e}$ corresponds to an argument to \texttt{Fun}. 

Ideally, $\bbracket{-}$ should be type-preserving -- well-typed source programs are translated to well-typed target programs. Clearly, the hypothesis of having a type-preserving transformation where $\bbracket{\pitype{x}{A}{B}} =$ \texttt{Pi }$\bbracket{A}\ \bbracket{B}$ does not hold, since it translates functions of type $A \rightarrow (A \rightarrow A)$ to constructors returning \texttt{Pi} $\bbracket{A}$ \texttt{(Pi} $\bbracket{A}$ $\bbracket{A}$ \texttt{)}, which fails the positivity check for inductive definitions. In fact, we cannot translate $\pitype{x}{A}{B}$ to an element of any inductive family in a type-preserving way. I proof this with the help of the following lemma.

\begin{lemma} (Universe preservation) Type-preserving transformation $\bbracket{-}$ turns a universe in CC to a universe in Agda and it preserves the universe hierarchy. 
\paragraph{Proof} 
We have $\cdot \vdash 0 : Nat : U_0 : U_1 : \cdots$ in CC (with freely-introduced natural numbers as a base type). To satisfy type preservation, we must have $\bbracket{0} : \bbracket{Nat} : \bbracket{U_0}: \bbracket{U_1} : \cdots$ in the target language. 
Therefore, $\bbracket{U_0}, \bbracket{U_1}, \cdots$ are universes (since they are the types of types) and they form a universe hierarchy in the target language. \qed
\label{lemma : universe}
\end{lemma}
By universe preservation, for all CC expression $e_1, e_2$, if $e_1$ is in a higher universe than $e_2$:
\begin{align*}
  & \Gamma \vdash e_1 : A, \quad \Gamma \vdash A : U_m, \\
  & \Gamma \vdash e_2 : B, \quad \Gamma \vdash B : U_n \ (m > n),
\end{align*}
then $\bbracket{e_1}$ will also be in a higher universe than $\bbracket{e_2}$ in Agda.
\begin{align*}
  & \bbracket{e_1} : \bbracket{A}, \quad \bbracket{A} : \mathsf{Set_i},\\
  & \bbracket{e_2} : \bbracket{B}, \quad \bbracket{B} : \mathsf{Set_j}\ (\mathsf{i > j}).
\end{align*}
Now, I proof that type-preserving defunctionalization cannot be defined using inductive families.

\begin{theorem} If $\bbracket{-}$ is type-preserving, then $\bbracket{\pitype{x}{A}{B}}$ cannot be an element of any inductive family.
\paragraph{Proof} 
Assume that $\bbracket{-}$ is type-preserving, $\bbracket{\pitype{x}{A}{B}}$ is an element of an inductive family \texttt{D}, and $\Gamma \vdash \pitype{x}{A}{B} : U_m$. Assume $\bbracket{U_m} =$ \codemath{Set_n}, which is reasonable by Lemma \ref{lemma : universe}. Data type \texttt{D} must be defined in the following form:
\begin{lstlisting}[escapechar=@]
data D : (@\codemath{y_1}@ : @\codemath{T_1}@) -> @\codedots@ -> (@\codemath{y_n}@ : @\codemath{T_n}@) -> @\codemath{Set_n}@
\end{lstlisting}
\texttt{D} is in universe \codemath{Set_n} because $\bbracket{-}$ is type preserving.
Pick an arbitrary function $f$ in the source language and suppose that its type is $\pitype{x}{A}{B}$. By assumption, $\bbracket{f} =$ \texttt{Fun}, which is a valid constructor of \texttt{D}. The general form of \texttt{Fun} is
\begin{lstlisting}[escapechar=@]
Fun : (@\codemath{z_1}@ : @\codemath{S_1}@) -> @\codedots@ -> (@\codemath{z_m}@ : @\codemath{S_m}@) -> (D @\codemath{t_1}@ @\codedots@ @\codemath{t_n}@)
\end{lstlisting}
where each \codemath{(z_i : S_i)} corresponds to a free variable in $f$. Since \texttt{Fun} is a valid constructor, the universe of each \codemath{S_i} must be smaller than \codemath{Set_n} to pass the universe check. However, this is not true in general: $f$ may contain free variables from universes higher than $U_m$, which are translated to terms in universes higher than \codemath{Set_n}, by Lemma \ref{lemma : universe}. Therefore, we have a contradiction. $\qed$
\label{theorem: dfc}
\end{theorem}

A type-preserving transformation cannot take $\bbracket{\pitype{x}{A}{B}} =$ \texttt{Pi }$\bbracket{A}\ \bbracket{B}$, as results of this translation fail the positivity check. Theorem \ref{theorem: dfc} shows that no inductive family can be used for type-preserving defunctionalization, and the key problem is the universe check that forbids impredicative constructors. If a function contains free variables in universes higher than the universe of the function's type, this function will be translated to a constructor that takes arguments in universes higher than the universe of its inductive family, which fails the universe check. Previous defunctionalization methods relied on the fact that any type can be an argument to a constructor, which is impredicative in nature.




% !TEX root = ../main.tex

\section{Main ideas}

Abstractly speaking, defunctionalization is a transformation that eliminates functions by replacing them with a special kind of first-class values, which I refer to as \textit{labels}. Each source-program function uniquely corresponds to a label in the target program. 
A label carries a list of values assigned to its corresponding function's free variables, similar to a function closure with the closure environment. 
The transformation turns applications of functions to arguments into applications of labels to (transformed) arguments, and it provides a mechanism for evaluating label applications.

Concretely, the polymorphic defunctionalization in OCaml uses constructors of a generalized abstract data type (GADT) \texttt{lam} as labels, and the mechanism for evaluating label applications is an auxiliary \textit{apply} function. Chapter 3 discussed an attempt of adapting this method to the dependently-typed world using inductive families (the counterpart of GADTs in dependently-typed languages) as labels. 
This turned out to be a failure, because constructors of inductive families cannot accept arguments from universes higher than the universe of the inductive definition, but functions could contain free variables from arbitrarily high universes.

One alternative approach is to design a target language with a primitive notion of labels that fits the abstract description of defunctionalization. Defining abstract transformations with specialized target languages is a common approach in the literature. For example, Minamide et al.~presented a type-preserving closure conversion from the simply-typed lambda calculus to a typed intermediate language with closures as first-class values~\cite{DBLP:conf/popl/MinamideMH96}. They named this method \textit{abstract closure conversion}, and Bowman et al.~adapted it to dependently-typed closure conversion~\cite{DBLP:conf/pldi/BowmanA18}. 

Abstract defunctionalization took inspiration from works of Minamide et al. and Bowman et al. It consists of a target language, the Defunctionalized Calculus of Constructions (DCC), and a type-preserving transformation from CC into it. 
Before formally presenting DCC's syntax and type judgements, I give an informal explaination of its key features and the intuition behind. I write DCC expressions in a \color{blue!95!white}{\textsf{san-serif blue font}} \color{black} to distinguish them from source language expressions.

DCC is a language similar to CC, except that its syntax contains first-class function labels instead of lambda abstractions. The syntax of a label $\target{\flabel}\{\target{e_1} \sfcomma \cdots \sfcomma \target{e_n}\}$ consists of a label name ($\target{\flabel}$) and a list of free-variable values it carries ($\{\target{e_1} \sfcomma \cdots \sfcomma \target{e_n}\}$). DCC also provides label contexts as the mechanism for evaluating label applications. These contexts contain elements in the following form.
\begin{equation*}
	\itemdef{\target{\flabel}}{\itemtype{x_1}{A_1} \sfcomma \cdots \sfcomma \itemtype{x_n}{A_n}}{\targetpi{x}{A}{B}}{\target{e}}
\end{equation*}
If $\target{\flabel}$ corresponds to a function $f$ in the source language, then $\{\itemtype{\bar{x}}{\bar{A}}\}$, $\targetpi{x}{A}{B}$, and $\target{e}$ correspond to the types of free variables in $f$, $f$'s type, and $f$'s body respectively. Applying a label to an argument evaluates to $\target{e}$ with all the free-variable values substituted in.

For example, the label context associated with the defunctionalized natural-number addition function 
$\lam{x}{Nat}{(\lam{y}{Nat}{x + y})}$ is the following.
\begin{align*}
	&\itemdef{\target{\flabel_1}}{\itemtype{\target{x}}{Nat}}{\targetpi{y}{Nat}{Nat}}{\target{x} + \target{y}}\sfcomma\\
	& \itemdef{\target{\flabel_0}}{}
	{\target{Pi}\textsf{(}\target{x}\sfcomma \target{Nat}\sfcomma \targetpi{y}{Nat}{Nat}\textsf{)}}
	{\targetlab{\flabel_1}{x}}
\end{align*}

Informally, this label context says that $\target{\flabel_0}$ corresponds to a function that takes no free variables and has the type $\pitype{x}{Nat}{\pitype{y}{Nat}{Nat}}$. Given an input $x$, it returns another function whose free variable is assigned with the value of $x$. The interpretation for $\target{\flabel_1}$ is similar.

The reader might find the resemblance between DCC and defunctionalized programs in OCaml. Intuitively, a label term $\target{\flabel}\{\target{e_1} \sfcomma \cdots \sfcomma \target{e_n}\}$ is just like a constructor and all the arguments it takes (\texttt{Fun e1} \codemath{\cdots} \texttt{en}), and the label context is like the pattern-matching cases of the \texttt{apply} function. The difference is that a function's type, the types of its free variables, and the function body are encoded implictly in the type of \texttt{Fun} and the definition of \texttt{apply}, whereas DCC places them explicitly in the label context. 





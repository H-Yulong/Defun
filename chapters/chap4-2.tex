% !TEX root = ../main.tex

\section{Defunctionalized Calculus of\\Constructions (DCC)}

This section presents the formal definition of Defunctionalized Calculus of Constructions.

\subsection{Syntax}
DCC is a language similar to CC, except that its syntax (Figure~\ref{fig:dcc syntax}) contains first-class function labels instead of lambda abstractions.
DCC has an infinite hierarchy of predicative universes $\target{U_i}$, and its term expressions are: variables $\target{x}$ (from an infinite set of variable names), universes $\target{U}$, $\Pi$-types $\targetpi{x}{A}{B}$, applications $\targetapp{e_1}{e_2}$, and labels $\targetlab{\flabel}{\bar{e}}$. I omit ground types for simplicity. Except for labels, all term expressions in DCC are the same as their counterparts in CC (written down in a different way).

A label expression $\targetlab{\flabel}{\bar{e}}$ is a label name $\target{\flabel}$ supplied with a list of free-variables values $\target{e_1}, \cdots, \target{e_n}$. Labels can take zero free variable, but a label name by itself is not a term. In other words, $\target{\flabel}$ is not a term, but $\targetlab{\flabel}{}$ is. I write $\target{\bar{e}}$ as an abbreviation of a list of zero or more expressions. Label names $\target{\flabel_1}, \target{\flabel_2}, \cdots$ come from an infinite set of names which is disjoint to the set of variable names, and I write labels in a different font from variables to emphasise this.

A DCC context $\targetcon$ consists of a context $\target{\Gamma}$ for types and a context $\target{\fdef}$ for label definitions. The type context is a list of variable-expression pairs. The label context is a list of label names $\target{\flabel}$ and their associated data 
$\textsf{(}\{\itemtype{\bar{x}}{\bar{A}}\}\textsf{,}\ \targetpi{x}{A}{B} \textsf{,}\ \target{e} \textsf{)}$. 
The first part $\{\itemtype{\bar{x}}{\bar{A}}\}$ records the types of free variables the label takes. 
The second part $\targetpi{x}{A}{B}$ specifies the type of the label, and the third part $\target{e}$ is what the label reduces to when it is applied to an argument.

I use $\itemtype{\bar{x}}{\bar{A}}$ as a syntactic sugar for $\itemtype{x_1}{A_1}, \cdots, \itemtype{x_n}{A_n}$. 
Note that $\target{A_{i+1}}$ could depend on $\target{x_1}, \cdots, \target{x_i}$, and the list of free variables can be empty. Variables $\target{\bar{x}}$ are bound to $\target{A}$, $\target{B}$, and $\target{e}$; the variable $\target{x}$ in the second part $\targetpi{x}{A}{B}$ is bound to \textit{both} $\target{B}$ and $\target{e}$. 

%Substitution
Substitutions for variables, universes, $\Pi$-types and applications in DCC follow the conventional definition. Substitutions for labels are:
\begin{equation*}
\targetlab{\flabel}{\bar{e}}\targetsub{e}{x} \triangleq \target{\flabel}\{\target{\bar{e}}\targetsub{e}{x}\},
\end{equation*}
where $\target{\bar{e}}\targetsub{e}{x}$ is a syntactic sugar for $\target{e_1}\targetsub{e}{x}, \cdots, \target{e_n}\targetsub{e}{x}$.

\begin{figure}[H]
	\renewcommand{\arraystretch}{1.3}
	\centering
	\begin{tabular}{l r l l}
		\textit{Universes}   & $\target{U} $       & ::= & $\target{U_i}$ \\
		\textit{Expressions} & $\target{e}$\sfcomma $\target{A}$\sfcomma $\target{B}$  & ::= & 
			$\target{x}$ $\ |\ $ $\target{U}$ $\ |\ $ $\targetpi{x}{A}{B}$ $\ |\ $ $\targetapp{e_1}{e_2}$ $\ |\ $ $\targetlab{\flabel}{\bar{e}}$\\
		\textit{Type contexts} & $\target{\Gamma}$ & ::= & $\target{\cdot}$ $\ |\ $ $\target{\Gamma}$\sfcomma $\itemtype{x}{A}$ \\
		\textit{Label contexts} & $\target{\fdef}$ & ::= & $\target{\cdot}$ $\ |\ $ 
			$\target{\fdef}$\sfcomma $\itemdef{\target{\flabel}}{\itemtype{\bar{x}}{\bar{A}}}{\targetpi{x}{A}{B}}{\target{e}}$\\
		\textit{DCC contexts} &$\targetcon$
	\end{tabular}

	\caption{DCC syntax}
    \label{fig:dcc syntax}
\end{figure}

\subsection{Type judgements}
% An expression $\target{e}$ is well-typed in $\targetcon$ if $\targetcon \vdash \itemtype{e}{A}$ is derivable for some $\target{A}$.

% Type judgement
DCC's type judgements are of the form $\targetcon \vdash \itemtype{e}{A}$, and
typing rules are given in Figure~\ref{fig:dcc typing}. Rules for variables, universes, $\Pi$-types, applications, and conversion are identical to their counterpart rules in CC, so I focus on the rule for labels. A label term $\targetlab{\flabel}{\bar{e}}$ is well-typed in $\targetcon$ if they satisfy the following conditions.
\begin{enumerate}
	\item The context $\targetcon$ is \textit{well-formed}.
	\item Label $\target{\flabel}$ is present in $\target{\fdef}$ and it associates with 
		$\textsf{(}\{\itemtype{\bar{x}}{\bar{A}}\}\textsf{,}\ \targetpi{x}{A}{B} \textsf{,}\ \target{e} \textsf{)}$.
	\item The length of the two lists $\target{\bar{e}}$ and $\itemtype{\bar{x}}{\bar{A}}$ are equal. 
	\item All expressions in $\target{\bar{e}}$ are well-typed, and their types match the specified types of free variables $\target{\bar{A}}$. 
\end{enumerate}
Specifically, condition (4) means:
\begin{align*}
	\targetcon &\vdash \itemtype{e_1}{A_1},\\
	\targetcon &\vdash \itemtype{e_2}{A_2}\targetsub{e_1}{x_1},\\
	&\cdots\\
	\targetcon &\vdash \itemtype{e_n}{A_n}[\target{e_1} \slash \target{x_1}, \cdots, \target{e_{n-1}} \slash \target{x_{n-1}}].
\end{align*}
Each $\target{A_{i+1}}$ depends on $\target{x_1}, \cdots, \target{x_i}$, so $\target{e_1}, \cdots, \target{e_i}$ need to be substituted in $\target{A_{i+1}}$ in the type judgement for $\target{e_{i+1}}$. The type of $\targetcon \vdash \itemtype{e}{A}$ is
\begin{equation*}
\target{Pi}\textsf{(}\target{x}\textsf{,} \target{A} \targetsub{\bar{e}}{\bar{x}}\textsf{,} \target{B} \targetsub{\bar{e}}{\bar{x}}\textsf{)}.
\end{equation*}
Note that values of free variables $\target{\bar{e}}$ are substituted in $\targetpi{x}{A}{B}$, the specified type of the label. I use $\targetsub{\bar{e}}{\bar{x}}$ is a syntactic sugar of $[\target{e_1}\slash\target{x_1}\sfcomma \cdots\sfcomma \target{e_n}\slash\target{x_n}]$, and conditions (3) and (4) are abbreviated to $\targetcon \vdash \itemtype{\bar{e}}{\bar{A}}$ as a convention.

% Well-formed context
The DCC judgement for well-formed contexts is $\vdash \targetcon$ and the judgement for well-formed label contexts is $\vdash \target{\fdef}$. Their derivation rules are given in Figure~\ref{fig:dcc context}. A well-formed label context with an empty type context is a well-formed DCC context (WF-Empty). 
If $\targetcon$ is well-formed and $\target{A}$ is a valid type in it, then $\target{\fdef} \semicolon \target{\Gamma} \sfcomma \itemtype{x}{A}$ is also well-formed (WF-Type). 

% Well-formed label context
For label contexts, the empty context is well-formed (WFL-Empty). A label context extended with a new entry $\itemdef{\target{\flabel}}{\itemtype{\bar{x}}{\bar{A}}}{\targetpi{x}{A}{B}}{\target{e}}$ is well-formed if $\targetpi{x}{A}{B}$ is a valid type in $\target{\fdef} \semicolon \target{\Gamma_{fv}}$ and the type of $\target{e}$ is $\target{B}$ in $\target{\fdef} \semicolon \target{\Gamma_{fv}} \sfcomma \itemtype{x}{A}$ (WFL-Label). Here, $\target{\Gamma_{fv}}$ is the \textit{free-variable type context} formed by the label's free variables, defined as $\target{\Gamma_{fv}} = \target{\cdot} \sfcomma \itemtype{\bar{x}}{\bar{A}}$. Both the type context and the label context have the weakening property -- a well-typed term is still well-typed in a extended type or label context.

\begin{lemma}[Type context weakening] If $\targetcon \vdash \itemtype{e}{A}$ and $\targetcon \vdash \itemtype{B}{U}$, then $\targetcon \sfcomma \itemtype{x}{B} \vdash \itemtype{e}{A}$.
\end{lemma}

\begin{lemma}[Label context weakening] If $\targetcon \vdash \itemtype{e}{C}$,
$\target{\fdef}\semicolon \target{\Gamma_{fv}} \vdash \targetpi{x}{A}{B} \mathrel{:} \target{U}$, and
$\target{\fdef}\semicolon \target{\Gamma_{fv}}\sfcomma \itemtype{x}{A} \vdash \itemtype{e'}{B}$, then
$\sfpl \target{\fdef}\sfcomma \itemdef{\target{\flabel}}{\itemtype{\bar{x}}{\bar{A}}}{\targetpi{x}{A}{B}}{\target{e'}} \sfpr 
\semicolon \target{\Gamma} \vdash \itemtype{e}{C}
$.
\end{lemma}

\begin{figure}
\renewcommand{\arraystretch}{1.3}
	\begin{equation}
		\tag{Var}
		\frac
			{\itemtype{x}{A} \in \target{\Gamma} \qquad
			 \vdash \target{\fdef}\semicolon \target{\Gamma} }
			{\target{\fdef}\semicolon \target{\Gamma} \vdash \itemtype{x}{A}}
	\end{equation}

	\begin{equation}
		\tag{Universe}
		\frac
			{\vdash \target{\fdef}\semicolon \target{\Gamma} }
			{\target{\fdef}\semicolon \target{\Gamma} \vdash \itemtype{U_i}{U_{i+1}}}
	\end{equation}

	\begin{equation}
		\tag{Pi}
		\frac
			{\target{\fdef}\semicolon \target{\Gamma} \vdash \itemtype{A}{U_i} \qquad
			 \target{\fdef}\semicolon \target{\Gamma}\sfcomma \itemtype{x}{A} \vdash \itemtype{B}{U_j}
			}
			{\target{\fdef}\semicolon \target{\Gamma} \vdash \targetpi{x}{A}{B} \mathrel{:} \target{U}_{\target{max}(\target{i},\ \target{j})} }
	\end{equation}

	\begin{equation}
		\tag{Apply}
		\frac
			{\target{\fdef}\semicolon \target{\Gamma} \vdash \target{e_1} \mathrel{:} \targetpi{x}{A}{B} \qquad
			 \target{\fdef}\semicolon \target{\Gamma} \vdash \itemtype{e_2}{A} }
			{\target{\fdef}\semicolon \target{\Gamma} \vdash \targetapp{e_1}{e_2} \mathrel{:} \target{B} \targetsub{e_2}{x}}
	\end{equation}

	\begin{equation}
		\tag{Label}
		\frac
			{\begin{array}{l}
			  	\vdash \target{\fdef}\semicolon \target{\Gamma} \qquad\qquad\qquad \target{\fdef}\semicolon \target{\Gamma} \vdash \itemtype{\bar{e}}{\bar{A}} \\
			  	\itemdef{\target{\flabel}}{\itemtype{\bar{x}}{\bar{A}}}{\targetpi{x}{A}{B}}{\target{e}} \in \target{\fdef} \\
			 \end{array}	 
			}
			{\quad \target{\fdef}\semicolon \target{\Gamma} \vdash \targetlab{\flabel}{\bar{e}} \mathrel{:} 
			 \target{Pi}\textsf{(}\target{x}\textsf{,} \target{A} \targetsub{\bar{e}}{\bar{x}}\textsf{,} \target{B} \targetsub{\bar{e}}{\bar{x}}\textsf{)} \quad
			}
	\end{equation}

	\begin{equation}
		\tag{Equiv}
		\frac
			{\target{\fdef}\semicolon \target{\Gamma} \vdash \itemtype{e}{A} \qquad
			 \target{\fdef}\semicolon \target{\Gamma} \vdash \itemtype{B}{U} \qquad
			 \target{\fdef}\semicolon \target{\Gamma} \vdash \target{A} \equiv \target{B}
			}
			{\target{\fdef}\semicolon \target{\Gamma} \vdash \itemtype{e}{B}}
	\end{equation}
	\caption{DCC Typing}
    \label{fig:dcc typing}
\end{figure}

\begin{figure}
\renewcommand{\arraystretch}{1.3}
	\begin{equation}
		\tag{WF-Empty}
		\frac{\vdash \target{\fdef} }{\quad \vdash \target{\fdef} \semicolon \target{\cdot} \quad}
	\end{equation}\vspace{0.1cm}

	\begin{equation}
		\tag{WF-Type}
		\frac{\target{\fdef}\semicolon \target{\Gamma} \vdash \itemtype{A}{U}}
		{\vdash \target{\fdef}\semicolon \target{\Gamma}\sfcomma \itemtype{x}{A}}
	\end{equation}

	\begin{equation}
		\tag{WFL-Empty}
		\frac{}{\quad \vdash \target{\cdot} \quad}
	\end{equation}
	
	\begin{multline}
		\tag{WFL-Label}
		\qquad\qquad\qquad
		\frac{
			\target{\fdef}\semicolon \target{\Gamma_{fv}} \vdash \targetpi{x}{A}{B} \mathrel{:} \target{U} \qquad
			\target{\fdef}\semicolon \target{\Gamma_{fv}}\sfcomma \itemtype{x}{A} \vdash \itemtype{e}{B}
			}
		{
			\vdash \target{\fdef}\sfcomma \itemdef{\target{\flabel}}{\itemtype{\bar{x}}{\bar{A}}}{\targetpi{x}{A}{B}}{\target{e}}
		}\\
		\text{where } \target{\Gamma_{fv}} \triangleq \target{\cdot}\sfcomma \itemtype{\bar{x}}{\bar{A}}
	\end{multline}
	\caption{Well-formedness of DCC contexts and label contexts}
    \label{fig:dcc context}
\end{figure}

\begin{figure}
\renewcommand{\arraystretch}{1.3}

	\begin{align*}
		\targetcon \vdash &\ \target{Apply\ } \target{\flabel}\{\target{\bar{e}}\}\ \target{e_2}\ \triangleright\  
		\target{e_1}[\target{\bar{e}} \slash \target{\bar{x}} \sfcomma \target{e_2} \slash \target{x}] \\
		& \text{where } \itemdef{\target{\flabel}}{\itemtype{\bar{x}}{\bar{A}}}{\targetpi{x}{A}{B}}{\target{e_1}} \in \target{\fdef}
	\end{align*}

	\begin{equation}
		\tag{Eq-eq}
		\frac
			{\target{\fdef}\semicolon \target{\Gamma} \vdash \target{e_1} \triangleright^* \target{e} \qquad 
			 \target{\fdef}\semicolon \target{\Gamma} \vdash \target{e_2} \triangleright^* \target{e}}
			{\target{\fdef}\semicolon \target{\Gamma} \vdash \target{e_1} \equiv \target{e_2}}
	\end{equation}

	\begin{equation}
		\tag{Eta1}
		\frac
			{\begin{array}{l}
			  \target{\fdef}\semicolon \target{\Gamma} \vdash \target{e_1} \triangleright^* \targetlab{\flabel}{\bar{e}} \qquad
	          \target{\fdef}\semicolon \target{\Gamma} \vdash \target{e_2} \triangleright^* \target{e_2'} \\
	          \itemdef{\target{\flabel}}{\itemtype{\bar{x}}{\bar{A}}}{\targetpi{x}{A}{B}}{\target{e}} \in \target{\fdef} \\
	          \target{\fdef}\semicolon \target{\Gamma}\sfcomma \itemtype{x}{A} \targetsub{\bar{e}}{\bar{x}} \vdash 
	          \target{e}\targetsub{\bar{e}}{\bar{x}} \equiv \targetapp{e_2'}{x}
        	\end{array}}
			{\target{\fdef}\semicolon \target{\Gamma} \vdash \target{e_1} \equiv \target{e_2}}
	\end{equation}

	\begin{equation}
		\tag{Eta2}
		\frac
			{\begin{array}{l}
			  \target{\fdef}\semicolon \target{\Gamma} \vdash \target{e_1} \triangleright^* \target{e_1'} \qquad
	          \target{\fdef}\semicolon \target{\Gamma} \vdash \target{e_2} \triangleright^* \targetlab{\flabel}{\bar{e}} \\
	          \itemdef{\target{\flabel}}{\itemtype{\bar{x}}{\bar{A}}}{\targetpi{x}{A}{B}}{\target{e}} \in \target{\fdef} \\
	          \target{\fdef}\semicolon \target{\Gamma}\sfcomma \itemtype{x}{A} \targetsub{\bar{e}}{\bar{x}} \vdash 
	          \targetapp{e_1'}{x} \equiv \target{e}\targetsub{\bar{e}}{\bar{x}}
        	\end{array}}
			{\target{\fdef}\semicolon \target{\Gamma} \vdash \target{e_1} \equiv \target{e_2}}
	\end{equation}

	\caption{DCC reduction and equivalence}
    \label{fig:dcc equivalence}
\end{figure}

% Reduction and equivalence
DCC's reductions $\targetcon \vdash \target{e_1} \triangleright \target{e_2}$ and equivalence $\targetcon \vdash \target{e_1} \equiv \target{e_2}$ are defined in Figure~\ref{fig:dcc equivalence}. 
There is only one reduction rule in DCC: $\target{Apply\ } \targetlab{\flabel}{\bar{e}}\ \target{e_2}$
reduces to $\target{e_1}[\target{\bar{e}} \slash \target{\bar{x}} \sfcomma \target{e_2} \slash \target{x}]$, where $\target{e_1}$ is found in the label's associated data $\itemdef{\target{\flabel}}{\itemtype{\bar{x}}{\bar{A}}}{\targetpi{x}{A}{B}}{\target{e_1}}$ in the label context $\target{\fdef}$. A reduction sequence is noted as $\target{e_1} \triangleright^* \target{e_2}$, which means $\target{e_1}$ reduces to $\target{e_2}$ in zero or more steps.

Two terms $\target{e_1}$ and $\target{e_2}$ are equivalent if they both reduce to the same term $\target{e}$ in a reduction sequence or they are $\eta$-equivalent. DCC's $\eta$-equivalence rules are similar to that of CC. 
Rule (Eta1) defines that $\target{e_1}$ and $\target{e_2}$ are equivalent if they satisfy the following conditions.
\begin{enumerate}
	\item  $\target{e_1}$ reduces to a label $\targetlab{\flabel}{\bar{e}}$ in a reduction sequence.
	\item  $\target{e_2}$ reduces to some expression $\target{e_2'}$ in a reduction sequence.
	\item  $\target{\flabel}$ associates with 
	$\textsf{(}\{\itemtype{\bar{x}}{\bar{A}}\}\textsf{,}\ \targetpi{x}{A}{B} \textsf{,}\ \target{e} \textsf{)}$ 
	in the label context $\target{\fdef}$.
	\item  $\targetapp{e_2'}{x}$ is equivalent to $\target{e}\targetsub{\bar{e}}{\bar{x}}$ in $\targetcon\sfcomma\itemtype{x}{A}$.
\end{enumerate}
Rules (Eta1) and (Eta2) are symmetrical.


% Informally, an expression $\target{e}$ is equivalent to a label $\targetdef{\flabel}{\bar{e}}$ if, for all expressions arg, Apply L{e-} arg \equiv Apply e arg.


% \subsection{Properties}

% \begin{lemma}[Type context weakening] If $\target{\fdef}\semicolon \target{\Gamma} \vdash \itemtype{e}{A}$, $\target{\fdef}\semicolon \target{\Gamma} \vdash \itemtype{B}{U}$, and $\itemtype{x}{B} \notin \target{\Gamma}$, then $\target{\fdef}\semicolon \target{\Gamma}, \itemtype{x}{B} \vdash \itemtype{e}{A}$.

% \end{lemma}

% \begin{lemma}[Function context weakening] If $\target{\fdef_1}\semicolon \target{\Gamma} \vdash \itemtype{e}{A}$ and $\vdash \target{\fdef_2} \semicolon \target{\Gamma}$, then $\target{\fdef_1} \cup \target{\fdef_2}\semicolon \target{\Gamma} \vdash \itemtype{e}{A}$.
% \end{lemma}
% !TEX root = ../main.tex



\section{Transformation}

This section gives the definition of dependently typed defunctionalization transformation.
The transformation consists of two parts: a transformation $\bbracket{-}$ for terms and a metafunction $\bbracket{-}_d$ that extracts function definitions from the source program. The term transformation produces the target program and the metafunction $\bbracket{-}_d$ gives a label context, which functions similarly to the auxiliary function \textit{apply} in polymorphic defunctionalization.

\subsection{Term transformation}

The term transformation $\bbracket{-}$ turns expressions in CC into expressions in DCC. The transformation is type-driven, i.e.~it $\bbracket{e}$ takes $\Gamma \vdash e \goodcolon A$ (the type judgement of $e$) as an implicit argument.

\begin{definition}	
The term transformation of a CC-expression $\bbracket{e}$ is $\target{e}$, where $\Gamma \vdash e \goodcolon A \leadsto \target{e}$. The wavy arrow $\leadsto$ is a relation between type judgements in CC and terms in DCC  (Figure~\ref{fig:dcc transformation}), defined inductively on CC's type derivation rules.
\end{definition}

\begin{figure}[H]
	\begin{equation}
		\tag{T-Var}
		\frac
			{}
			{\Gamma \vdash x \goodcolon A \leadsto \target{x}}
	\end{equation} \hspace{0.5cm}
	\begin{equation}
		\tag{T-Universe}
		\frac
			{}
			{\Gamma \vdash U_i \goodcolon U_{i+1} \leadsto \target{U_i}}
	\end{equation} \hspace{0.5cm}
	\begin{equation}
		\tag{T-Pi}
		\frac
			{\Gamma \vdash A \goodcolon U_i \leadsto \target{A} \qquad \Gamma, x \goodcolon A \vdash B \goodcolon U_j \leadsto \target{B}
			}
			{\Gamma \vdash \pitype{x}{A}{B} \mathrel{:} U_{max(i, j)} \leadsto \targetpi{x}{A}{B}}
	\end{equation} \hspace{0.5cm}
	\begin{equation}
		\tag{T-Apply}
		\qquad\qquad\quad
		\frac
			{\Gamma \vdash e_1 \goodcolon \pitype{x}{A}{B} \leadsto \target{e_1} \qquad \Gamma \vdash e_2 \goodcolon A \leadsto \target{e_2}}
			{\Gamma \vdash e_1 \ e_2 \mathrel{:} B\sub{e2}{x} \leadsto \targetapp{e_1}{e_2}}
	\end{equation} \hspace{0.5cm}
	\begin{equation}
		\tag{T-Lambda}
		\qquad\qquad\qquad
		\frac
			{ \text{FV}(\lami{i}{x}{A}{e}) = \bar{x} \goodcolon \bar{A} \qquad \Gamma \vdash \bar{x} \goodcolon \bar{A} \leadsto \target{\bar{x}} }
			{ \Gamma \vdash \lami{i}{x}{A}{e} \mathrel{:} \pitype{x}{A}{B} \leadsto \target{\flabel_i}\{\target{\bar{x}}\}}
	\end{equation} \hspace{0.5cm}
	\begin{equation}
		\tag{T-Equiv}
		\frac
			{\Gamma \vdash e \goodcolon A \leadsto \target{e}}
			{\Gamma \vdash e \goodcolon B \leadsto \target{e}}
	\end{equation}
	\caption{Defunctionalization transformation}
    \label{fig:dcc transformation}
\end{figure}

The term transformation is straightforward. It turns source-language variables $x$ and universes $U_i$ into target-language variables $\target{x}$ and universes $\target{U_i}$, and recursively transforms $\Pi$-types $\pitype{x}{A}{B}$ and applications $e_1\ e_2$ into $\targetpi{x}{A}{B}$ and $\targetapp{e_1}{e_2}$.

Functions in the source language are translated to labels in the target language.
Defunctionalization requires a unique correspondence between each label and each source-program function.
I use a convention that every function in the transformation's input $e$ is tagged with a unique identifier $i$ ($i \in \mathbb{N}$), and its corresponding label's name is $\target{\flabel_i}$.

The transformation turns a function $(\lami{i}{x}{A}{e})$ into a label $\targetlab{\flabel_i}{\bar{x}}$, where $\target{\bar{x}}$ come from the function's free variables $\bar{x}$ (T-Lambda). The metafunction FV (see Definition~\ref{def:FV}) computes all free variables and their types involved in a well-typed CC-expression. Note that FV is different from \textit{fv}, the conventional free variable function that computes all the \textit{unbound variables} in an expression. In dependently typed languages, the type of a free variable may contain other free variables, and their types may still contain other free variables, and so on! Therefore, FV$(e)$ should recursively work out all the variables needed for $e$ to be well-typed. 

\begin{definition}FV$(e)$ takes $\Gamma \vdash e \goodcolon A$, the type judgement of $e$, as an implicit argument. It firstly computes all the unbound variables $x_1, \cdots, x_n$ in $e$ and in $A$, then calls itself recursively on types of these variables, and finally returns the union of all free variables and their types it found.
\begin{equation*}
\renewcommand{\arraystretch}{1.3}
\begin{array}{l l l}
	\text{FV}(e) & = & \text{FV}(A_1) \cup \cdots \cup \text{FV}(A_n) \cup \Gamma_{fv}\\
	& \text{where} & \textit{fv }(e) \cup \textit{fv }(A) = x_1, \cdots, x_n\\
	& & \Gamma \vdash x_1 \goodcolon A_1, \cdots, \Gamma \vdash x_n \goodcolon A_n\\
	& & \Gamma_{fv} \triangleq x_1 \goodcolon A_1, \cdots, x_n \goodcolon A_n.\\
\end{array}
\end{equation*}
\label{def:FV}
\end{definition}

\begin{lemma}If $\Gamma \vdash e \goodcolon A$, then $\text{FV}(e) \vdash e \goodcolon A$.
\label{lem:fv}
\end{lemma}

Since FV($e$) gives all the variables needed to correctly type $e$, if $e$ is well-typed in some context $\Gamma$, it is also well-typed in FV($e$). The formal proof of this lemma can be found in the appendix.

\subsection{Extracting function definitions}

Defunctionalization is not complete with just the transformation -- it turns functions into labels but throws away the function body. The metafunction $\bbracket{-}_d$ takes a CC-expression and returns a label context $\target{\fdef}$.
For every function $(\lami{i}{x}{A}{e})$ in the source program, the following item is in $\target{\fdef}$, where $\{\itemtype{\bar{x}}{\bar{A}}\}$, $\targetpi{x}{A}{B}$, and $\target{e}$ correspond to the free variables $\bar{x} \mathrel{:} \bar{A}$ in $\lambda^i$, the type of $\lambda^i$ and the function body $e$ respectively.
\begin{equation*}
	\itemdef{\target{\flabel_i}}{\itemtype{\bar{x}}{\bar{A}}}{\targetpi{x}{A}{B}}{\target{e}}
\end{equation*}


Function extraction is more subtle than transforming terms. Functions may appear in the type of an expression, even if the expression itself does not contain that function! For example, consider the following CC context and expression (with base type natural numbers $\mathbb{N}$).

\begin{align*}
	\Gamma\ & \triangleq\ \cdot,\ A \goodcolon (\mathbb{N} \rightarrow \mathbb{N}) \rightarrow U_0,\ 
	a \goodcolon \pitype{f}{(\mathbb{N} \rightarrow \mathbb{N})}{A\ (\lam{n}{\mathbb{N}}{(f\ n) + 1})} \\
	e\ & \triangleq\ a\ (\lam{x}{\mathbb{N}}{x+1})
\end{align*}

$A$ is a family of types indexed by functions of type $\mathbb{N} \rightarrow \mathbb{N}$, and $a\ f$ constructs an element of type $A\ (\lam{n}{\mathbb{N}}{(f\ n) + 1})$. According to the rule (Apply), the type of $e$ is 
\begin{align*}
	& (A\ (\lam{n}{\mathbb{N}}{(f\ n) + 1}))\sub{(\lam{x}{\mathbb{N}}{x+1})}{f}\\
	=& A\ (\lam{n}{\mathbb{N}}{((\lam{x}{\mathbb{N}}{x+1})\ n) + 1}).
\end{align*}
Hence, a new function definition appeared in the type of $e$ as the result of a substitution! This new function should be included in the label context $\target{\fdef}$ to achieve type preservation. Extracting function definitions in $e$ involes finding every function that appeared in the type derivation of $e$. In other words, the transformation defunctionalizes not just the source-language expression, but its entire type derivation tree.

\begin{definition}
The function definitions extracted from a CC-expression $\bbracket{e}_d$ is a label context $\target{\fdef}$, where 
$\Gamma \vdash e \goodcolon A \leadsto_d \target{\fdef}$. Figure~\ref{fig:dcc def} defines the relation $\leadsto_d$ between type judgements in CC and label contexts in DCC.
\end{definition}

\begin{figure}[H]
	\begin{equation}
		\tag{D-Var}
		\frac
			{\Gamma \vdash A \goodcolon U \leadsto_d \target{\fdef}}
			{\Gamma \vdash x \goodcolon A \leadsto_d \target{\fdef}}
	\end{equation} \vspace{0.5cm}
	\begin{equation}
		\tag{D-Universe}
		\frac
			{}
			{\Gamma \vdash U_i \goodcolon U_{i+1} \leadsto_d \target{\cdot}}
	\end{equation} \vspace{0.5cm}
	\begin{equation}
		\tag{D-Pi}
		\frac
			{\Gamma \vdash A \goodcolon U_i \leadsto_d \target{\fdef_A} \qquad \Gamma, x \goodcolon A \vdash B \goodcolon U_j \leadsto_d \target{\fdef_B}
			}
			{\Gamma \vdash \pitype{x}{A}{B} \mathrel{:} U_{max(i, j)} \leadsto_d \target{\fdef_A} \cup \target{\fdef_B}}
	\end{equation} \vspace{0.5cm}
	\begin{equation}
		\tag{D-Apply}
		\qquad\qquad\quad
		\frac{{\begin{array}{c c}
			  {\Gamma \vdash e_1 \mathrel{:} \pitype{x}{A}{B} \leadsto_d \target{\fdef_1} \qquad \Gamma \vdash e_2 \goodcolon A \leadsto_d \target{\fdef_2}} \\
	           \Gamma \vdash B\sub{e2}{x} \mathrel{:} U \leadsto_d \target{\fdef_3}\\
        	\end{array}}}
			{\Gamma \vdash e_1 \ e_2 \mathrel{:} B\sub{e2}{x} \leadsto_d \target{\fdef_1} \cup \target{\fdef_2} \cup \target{\fdef_3}}
	\end{equation} \vspace{0.5cm}
	\begin{equation}
		\tag{D-Lambda}
		\qquad\ \ 
		\frac{{\begin{array}{l @{\qquad} l}
			  \Gamma \vdash A \goodcolon U \leadsto_d \target{\fdef_A} & \Gamma, x \goodcolon A \vdash e \goodcolon B \leadsto_d \target{\fdef_e} \\
			   \text{FV}(\lami{i}{x}{A}{e}) = \bar{x} \goodcolon \bar{A} &
			   \Gamma \vdash \bar{x} \goodcolon \bar{A} \leadsto \itemtype{\bar{x}}{\bar{A}} \\
			   \Gamma \vdash \pitype{x}{A}{B} \leadsto \targetpi{x}{A}{B} &
			   \Gamma, x \goodcolon A \vdash e \leadsto \target{e}\\
        	\end{array}}}
			{\begin{array}{c}
			 \Gamma \vdash \lami{i}{x}{A}{e} \mathrel{:} \pitype{x}{A}{B} \leadsto_d \\
			  \qquad \qquad \qquad \target{\fdef_A} \cup \target{\fdef_e} \sfcomma \itemdef{\target{\flabel_i}}{\itemtype{\bar{x}}{\bar{A}}}{\targetpi{x}{A}{B}}{\target{e}}
			 \end{array}
			}
	\end{equation} \vspace{0.5cm}
	\begin{equation}
		\tag{D-Equiv}
		\qquad\qquad
		\frac
			{\Gamma \vdash e \goodcolon A \leadsto_d \target{\fdef} \qquad \Gamma \vdash B \goodcolon U \leadsto_d \target{\fdef_B}}
			{\Gamma \vdash e \goodcolon B \leadsto_d \target{\fdef} \cup \target{\fdef_B}}
	\end{equation}
	\caption{Extracting function definitions}
    \label{fig:dcc def}
\end{figure}

Variables does not contain function definitions, so the definitions in $x$ are just the definitions in its type $A$ (D-Var). Universes do not contain functions. Definitions in a dependent function type $\pitype{x}{A}{B}$ are the union of definitions in $A$ and $B$ (D-Pi). 

Definitions in an application $e_1\ e_2$ are the union of definitions in $e_1$, $e_2$, and $B\sub{e_2}{x}$, since substitution creates new function definitions (D-Apply). For a lambda abstraction $\lami{i}{x}{A}{e}$, the definitions it contains are the union of definitions in $e$ and $A$, appended with $\target{\flabel_i}$, the definition of itself (D-Lambda).
If $e$ has type $B$ by the conversion rule, then the definitions involved in the derivation of $\Gamma \vdash e \goodcolon B$ are the union of definitions in the derivation of $\Gamma \vdash e \goodcolon A$ and definitions in $B$ (D-Equiv).

\begin{lemma} For all well-typed terms $\Gamma \vdash e \goodcolon A$ in CC, $\bbracket{A}_d \subseteq \bbracket{e}_d$.
\label{lem:subset}
\end{lemma}
\paragraph{Proof} I proof this lemma with a rule induction on rules defined in Figure~\ref{fig:dcc def}. 
I only show for case (D-Lambda), since all other cases are either trivially true or follows simply from the definition.

My goal in case (D-Lambda) is to show that $\bbracket{\pitype{x}{A}{B}}_d \subseteq \bbracket{\lami{i}{x}{A}{e}}_d$, and I have
\begin{equation*}
\Gamma \vdash \lami{i}{x}{A}{e} \goodcolon \pitype{x}{A}{B}\ \leadsto_d\ \target{\fdef_A} \cup \target{\fdef_e} \sfcomma \itemdef{\target{\flabel_i}}{\itemtype{\bar{x}}{\bar{A}}}{\targetpi{x}{A}{B}}{\target{e}}.
\end{equation*}
In other words, $\bbracket{\lami{i}{x}{A}{e}}_d = \bbracket{A}_d \cup \bbracket{e}_d$ plus the definition of $\lambda^i$. By definition, $\bbracket{\pitype{x}{A}{B}}_d\ = \bbracket{A}_d \cup \bbracket{B}_d$, and I have $\bbracket{A}_d \subseteq \bbracket{\lami{i}{x}{A}{e}}_d$. 
I have $\bbracket{B}_d \subseteq \bbracket{e}_d$ by the induction hypothesis, since $\Gamma, x:A \vdash e \goodcolon B$. Therefore, $\bbracket{B}_d \subseteq \bbracket{\lami{i}{x}{A}{e}}_d$. \qed

Finally, I show a worked example of dependently typed defunctionalization with informal explanations.

\begin{exmp}Let the source program be the polymorphic identity function in CC (see Example 2.2.1), and the lambda abstrctions are tagged with integers.

\begin{equation*}
	\cdot \vdash \lami{0}{A}{U_0}{(\lami{1}{x}{A}{x})} \mathrel{:} \pitype{A}{U_0}{A \rightarrow A}
\end{equation*}

The source program is a function with no free variables, so the term transformation simply turns it into a label supplied with nothing.

\begin{equation*}
	\bbracket{\lami{0}{A}{U_0}{(\lami{1}{x}{A}{x})}} = \targetlab{\flabel_0}{}
\end{equation*}

There are two function definitions ($\lambda^0$ and $\lambda^1$) in the derivation tree of the source program. 
Function $\lambda^1$ has one free variable $A$ of type $U_0$; its type is $A \rightarrow A$ and its function body is $x$. 
Function $\lambda^0$ has no free variable; its type is $\pitype{A}{U_0}{A \rightarrow A}$ and its function body is $\lambda^1$ where $A$ is free in it. Therefore, $\bbracket{\lami{0}{A}{U_0}{(\lami{1}{x}{A}{x})}}_d$ returns the following label context with two items.
\begin{align*}
	\target{\fdef} = \target{\cdot}\ \sfcomma\ &\itemdef{\target{\flabel_1}}{\itemtype{A}{U_0}}{\targetpi{x}{A}{A}}{\target{x}}\sfcomma\\
	& \itemdef{\target{\flabel_0}}{}
	{\target{Pi}\textsf{(}\target{A}\sfcomma \target{U_0}\sfcomma \targetpi{x}{A}{A}\textsf{)}}
	{\targetlab{\flabel_1}{A}}
\end{align*}

\end{exmp}





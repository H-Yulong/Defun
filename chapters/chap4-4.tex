% !TEX root = ../main.tex

Function extraction is more subtle than transforming terms. Functions may appear in the type of an expression, even if the expression itself does not contain that function! For example, consider the following CC context and expression (with base type natural numbers $\mathbb{N}$).

\begin{align*}
	\Gamma\ & \triangleq\ \cdot,\ A \goodcolon (\mathbb{N} \rightarrow \mathbb{N}) \rightarrow U_0,\ 
	a \goodcolon \pitype{f}{(\mathbb{N} \rightarrow \mathbb{N})}{A\ (\lam{n}{\mathbb{N}}{(f\ n) + 1})} \\
	e\ & \triangleq\ a\ (\lam{x}{\mathbb{N}}{x+1})
\end{align*}

$A$ is a family of types indexed by functions of type $\mathbb{N} \rightarrow \mathbb{N}$, and $a\ f$ constructs an element of type $A\ (\lam{n}{\mathbb{N}}{(f\ n) + 1})$. According to the rule (Apply), the type of $e$ is 
\begin{align*}
	& (A\ (\lam{n}{\mathbb{N}}{(f\ n) + 1}))\sub{(\lam{x}{\mathbb{N}}{x+1})}{f}\\
	=& A\ (\lam{n}{\mathbb{N}}{((\lam{x}{\mathbb{N}}{x+1})\ n) + 1}).
\end{align*}
Hence, a new function definition appeared in the type of $e$ as the result of a substitution! This new function should be included in the label context $\target{\fdef}$ to achieve type preservation. Extracting function definitions in $e$ involes finding every function that appeared in the type derivation of $e$. In other words, the transformation defunctionalizes not just the source-language expression, but its entire type derivation tree.


\section{Type preservation}

This section proves that the dependently typed defunctionalization is type-preserving and correct. The proof of type-preservation requires three lemmas: \textit{substitution}, \textit{preservation of reduction sequences}, and \textit{coherence}. The proof of correctness follows directly from the preservation of reduction sequences. Unless explicitly noted, I use $\target{e}$ to stand for $\bbracket{e}$ in this section.

\subsection{Substitution, reduction and coherence}

I start with the proof of the substitution lemma. It states that the transformation is compatible with substitutions, i.e.~$\bbracket{e_1 \sub{e_2}{x}} \equiv \target{e_1} \sub{\target{e_2}}{\target{x}}$. This equivalence is only valid in a \textit{suitable context} $\targetcon$ where $\target{\fdef}$ contains all the function definitions appeared in $e_1$, $e_2$, and $e_1\sub{e_2}{x}$. Having a suitable context is necessary for establishing the  $\eta$-equivalence between two labels -- they cannot be equivalent if one of them cannot be found in the label context!

Generally, I define the suitable contexts for a theorem (or lemma) as the DCC contexts containing all the function definitions in all expressions mentioned in the theorem. The other two lemmas have their version of suitable contexts, and I omit their explicit definitions for simplicity.

\begin{lemma}[Substitution] For all well-typed substitutions $e_1 \sub{e_2}{x}$,\\
$\bbracket{e_1 \sub{e_2}{x}} \equiv \target{e_1}\sub{\target{e_2}}{\target{x}}$ under a suitable context for the substitution.

\label{lem:sub}
\end{lemma}
\paragraph{Proof.} I proof by induction on the syntax structure of $e_1$, and the induction hypothesis (IH) is $\bbracket{e_1 \sub{e_2}{x}} \equiv \target{e_1}\sub{\target{e_2}}{\target{x}}$. The most difficult case is when $e_1 = (\lami{i}{x}{A}{e})$, since the result of $e_1 \sub{e_2}{x}$ is a new function and the proof involves showing the $\eta$-equivalence of two labels.
\renewcommand{\arraystretch}{1.35}
\begin{longtable}{p{0.05\linewidth} p{0.95\linewidth}}
Case & $e_1 = y$: \\
& If $x = y$, then $\bbracket{x \sub{e_2}{x}} = \target{e_2} = \target{x}\sub{\target{e_2}}{\target{x}}$.\\
& If $x \neq y$, then $\bbracket{y \sub{e_2}{x}} = \target{y} = \target{y}\sub{\target{e_2}}{\target{x}}$. \\
\\

Case & $e_1 = U_i$:\\
& Trivial.\\
\\

Case & $e_1 = \pitype{y}{A}{B}$:\\
& $\bbracket{(\pitype{y}{A}{B})\sub{e_2}{x}} = \bbracket{\pitype{y}{A\sub{e_2}{x}}{B\sub{e_2}{x}}}$ by def.\\ 
& $\phantom{\bbracket{(\pitype{y}{A}{B})\sub{e_2}{x}}} = \target{Pi} \sfpl \target{y} \sfcomma \bbracket{A\sub{e_2}{x}} \sfcomma \bbracket{B\sub{e_2}{x}} \sfpr $ by def.\\
& $\phantom{\bbracket{(\pitype{y}{A}{B})\sub{e_2}{x}}} \equiv \target{Pi} \sfpl \target{y} \sfcomma \target{A}\sub{\target{e_2}}{\target{x}} \sfcomma \target{B}\sub{\target{e_2}}{\target{x}} \sfpr  $ by IH.\\
& $\phantom{\bbracket{(\pitype{y}{A}{B})\sub{e_2}{x}}} = \sfpl\target{Pi} \sfpl \target{y} \sfcomma \target{A} \sfcomma \target{B} \sfpr \sfpr \sub{\target{e_2}}{\target{x}} $ by def.\\

Case & $e_1 = e\ e'$:\\
& Similar to the previous case.\\
\\

Case & $e_1 = (\lami{i}{y}{A}{e})$:\\
& Without loss of generality, I assume that the type of $e_1$ is $\pitype{y}{A}{B}$, the type of $e_2$ is $C$, and FV($e_1$) $ =y_1 : A_1,\ \cdots,\ y_m : A_m,\ x : C,\ z_1 : B_1,\ \cdots,\ z_n : B_n$.
I abbreviate the free variables to $\bar{y} : \bar{A}, x : C , \bar{z} : \bar{B}$. Note that $x$ is free in $\bar{B}, A, B, \text{and } e$.
\\
& By definition, I have $\bbracket{(\lami{i}{y}{A}{e})\sub{e_2}{x}} = \bbracket{(\lami{j}{y}{A\sub{e_2}{x}}{e\sub{e_2}{x}})}$, where $j$  is a fresh tag.\\
& By term transformation, $\bbracket{(\lami{i}{y}{A}{e})} = 
\target{\flabel_i}\{\target{\bar{y}} \sfcomma \target{x} \sfcomma \target{\bar{z}}\}$, and I have 
$\itemdef{\target{\flabel_i}}
{\itemtype{\bar{y}}{\bar{A}} \sfcomma \itemtype{x}{C} \sfcomma \itemtype{\bar{z}}{\bar{B}}}
{\targetpi{y}{A}{B}}{\target{e}} \in \target{\fdef}$ by assumption (since $\bbracket{(\lami{i}{x}{A}{e})}_d \in \target{\fdef}$).\\
& Similatly, $\bbracket{(\lami{j}{y}{A\sub{e_2}{x}}{e\sub{e_2}{x}})} = 
\target{\flabel_j}\{\target{\bar{y}} \sfcomma \target{\bar{z}}\}$, and\\ 
&  $\itemdef{\target{\flabel_j}}
{\itemtype{\bar{y}}{\bar{A}} \sfcomma \target{\bar{z}} \goodcolon \bbracket{\bar{B}\sub{e_2}{x}}}
{\target{Pi} \sfpl \target{y} \sfcomma \bbracket{{A}\sub{e_2}{x}} \sfcomma \bbracket{{B}\sub{e_2}{x}} \sfpr}
{\bbracket{{e}\sub{e_2}{x}}} \in \target{\fdef}$.\\
& Now, my goal is to show that 
$\sfpl \target{\flabel_i}\{\target{\bar{y}} \sfcomma \target{x} \sfcomma \target{\bar{z}}\} \sfpr \sub{\target{e_2}}{\target{x}} = \target{\flabel_i}\{\target{\bar{y}} \sfcomma \target{e_2} \sfcomma \target{\bar{z}}\} \equiv
\target{\flabel_j}\{\target{\bar{y}} \sfcomma \target{\bar{z}}\}$. By rule (Eta1), it is sufficient to show that\\
& $\targetcon \sfcomma \target{y} \goodcolon \target{A} \vdash 
\target{e}[\target{\bar{y}}\slash\target{\bar{y}} \sfcomma \target{e_2}\slash\target{x} \sfcomma \target{\bar{z}}\slash\target{\bar{z}}]
\equiv \target{Apply}\ \target{\flabel_j}\{\target{\bar{y}} \sfcomma \target{\bar{z}}\} \ \target{y}$.\\
& Indeed, $\target{e}[\target{\bar{y}}\slash\target{\bar{y}} \sfcomma \target{e_2}\slash\target{x} \sfcomma \target{\bar{z}}\slash\target{\bar{z}}] = \target{e}\sub{\target{e_2}}{\target{x}}$, and\\
& $\target{Apply}\ \target{\flabel_j}\{\target{\bar{y}} \sfcomma \target{\bar{z}}\} \ \target{y}\ 
\triangleright\  \bbracket{{e}\sub{e_2}{x}}[\target{\bar{y}}\slash\target{\bar{y}} \sfcomma \target{\bar{z}}\slash\target{\bar{z}}] = \bbracket{{e}\sub{e_2}{x}} \equiv \target{{e}\sub{\target{e_2}}{\target{x}}}. \qed
$
\end{longtable}

The next two lemmas shows that the transformation is also compatible with reductions. Preservation of small step reduction shows that if a well-typed term $e_1$ reduces to $e_2$ in one step, then $\target{e_1}$ reduces to a term $\target{e}$ that is equivalent to $\target{e_2}$ in a reduction sequence.

\begin{lemma}[Preservation of small step reduction] For all well-typed terms $e_1$, if $e_1\ \triangleright\ e_2$, then $\target{e_1} \triangleright^* \target{e}$ for some $\target{e}$ that is equivalent to $\target{e_2}$ in a suitable context for the reduction.
\paragraph{Proof.} By induction on CC's only reduction rule: $(\lami{i}{x}{A}{e_1})\ e_2\ \triangleright\ e_1\sub{e_2}{x}$. My goal is to show that $\bbracket{(\lami{i}{x}{A}{e_1})\ e_2}\ \triangleright^*\ \target{e}$ for some $\target{e}$ that is equivalent to $\bbracket{e_1\sub{e_2}{x}}$.

Suppose the type of $(\lami{i}{x}{A}{e_1})$ is $\pitype{x}{A}{B}$. By term transformation, $\bbracket{(\lami{i}{x}{A}{e_1})} = \targetlab{\flabel_i}{\bar{x}}$, where $\target{\bar{x}}$ correspond to free variables $\bar{x}$ in $\lambda^i$. By assumption, $\itemdef{\target{\flabel_i}}{\itemtype{\bar{x}}{\bar{A}}}{\targetpi{x}{A}{B}}{\target{e_1}} \in \target{\fdef}$. So,
\begin{align*}
\bbracket{(\lami{i}{x}{A}{e_1})\ e_2} &= \target{Apply}\ \targetlab{\flabel_i}{\bar{x}}\ \target{e_2}\\
&\ \triangleright \target{e_1}[\target{\bar{x}}\slash\target{\bar{x}} \sfcomma \target{e_2}\slash\target{x}]\\
&= \target{e_1}\sub{\target{e_2}}{\target{x}}\\
&\equiv \bbracket{e_1\sub{e_2}{x}} \text{ by Lemma \ref{lem:sub}.} \qed
\end{align*}
\label{lem:small reduction}
\end{lemma}

The preservation of reduction sequences follows directly from Lemma~\ref{lem:small reduction}. 

\begin{lemma}[Preservation of reduction sequences] For all well-typed terms $e_1$, if $e_1\ \triangleright^* e_2$, then $\target{e_1} \triangleright^* \target{e}$ for some $\target{e}$ that is equivalent to $\target{e_2}$ in a suitable context for the reduction sequence.
\label{lem:reduction sequence}
\end{lemma}

\begin{coro}[Correctness] For all closed programs $\cdot \vdash e \goodcolon A$, if $A$ is a ground type and $e\ \triangleright^* v$ for some value $v$, then $\target{e}\ \triangleright^* \target{v}$ in $\bbracket{e}_d \semicolon \target{\cdot}$ .
\end{coro}

Correctness of the transformation is a special case of Lemma~\ref{lem:reduction sequence}.
Note that $\bbracket{A}_d \subseteq \bbracket{e}_d$ by Lemma~\ref{lem:subset} and ground type values do not contain function definitions, so $\bbracket{e}_d$ is a suitable context for applying the lemma.

Finally, I show that the transformation is compatible with the equivalence relation.

\begin{lemma}[Coherence] If $e_1 \equiv e_2$, then $\target{e_1} \equiv \target{e_2}$ in a suitable context for the equivalence.
\paragraph{Proof.} I prove by induction on CC's equivalence rules.
\renewcommand{\arraystretch}{1.35}
\begin{longtable}{p{0.05\linewidth} p{0.95\linewidth}}
Case & (Eq-eq): \\
& If $e_1 \equiv e_2$ by (Eq-eq) in CC, then $e_1\ \triangleright^* e$ and $e_2\ \triangleright^* e$ for some $e$. By preservation of reduction sequences, I have $\target{e_1}\ \triangleright^* \target{e_1'}$ where $\target{e_1'} \equiv \target{e}$ and $\target{e_2}\ \triangleright^* \target{e_2'}$ where $\target{e_2'} \equiv \target{e}$. 
Therefore, $\target{e_1} \equiv \target{e_1'}$ and $\target{e_2} \equiv \target{e_2'}$ by DCC's equivalence rule (Eq-eq), and $\target{e_1} \equiv \target{e_2}$ by the transitivity and the symmetry of the equivalence relation.
\\

Case & (Eq-Eta1):\\
& If $e_1 \equiv e_2$ by (Eq-Eta1) in CC, then $e_1\ \triangleright^* (\lami{i}{x}{A}{e})$, $e_2\ \triangleright^* e_2'$, and $e \equiv e_2'\ x$ (assuming that the type of $e_1$ is $\pitype{x}{A}{B}$). By preservation of reduction sequences, I have $\target{e_2}\ \triangleright^* \target{e_2''}$ where $\target{e_2''} \equiv \target{e_2'}$. By DCC's equivalence rule (Eq-eq) and the transitivity of the equivalence relation, $\target{e_2} \equiv \target{e_2'}$. Similarly, I have $\target{e_1} \equiv \targetlab{\target{\flabel_i}}{\bar{x}}$, where $\target{\bar{x}}$ correspond to the free variables $\bar{x}$ in $\lambda^i$. Also,
$\itemdef{\target{\flabel_i}}{\itemtype{\bar{x}}{\bar{A}}}{\targetpi{x}{A}{B}}{\target{e}} \in \target{\fdef}$.
\\
& Now my goal is to show that $\targetlab{\target{\flabel_i}}{\bar{x}} \equiv \target{e_2'}$. Indeed,
$\target{e}\sub{\target{\bar{x}}}{\target{\bar{x}}} = \target{e} \equiv \targetapp{e_2'}{x}$ by the inductive hypothesis and $e \equiv e_2'\ x$. Therefore, $\target{e_1} \equiv \target{e_2}$ by the transitivity and the symmetry of the equivalence relation.
\\
Case & (Eq-Eta2):\\
& This case is symmetric to case (Eq-Eta1). \qed
\end{longtable}
\label{lem:coherence}
\end{lemma}

\subsection{Type preservation}

Now, I can prove that the transformation is type-preserving by a simultaneous induction on the two mutually defined judgements in CC: the well-formedness of contexts $\vdash \Gamma$ and the type judgement $\Gamma \vdash e \goodcolon A$.

\begin{lemma}[Type preservation (technical)]
\label{lem:type-preservation-tech}
\end{lemma}

Finally, I present the type preservation theorem in a more desired form.

\begin{theorem}[Type preservation] For all type judgements $\Gamma \vdash e \goodcolon A$, $\bbracket{e}_d\ \semicolon \bbracket{\Gamma} \vdash \bbracket{e} \goodcolon \bbracket{A}$.
\paragraph{Proof}
\label{lem:type-preservation}
\end{theorem}


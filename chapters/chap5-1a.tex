% !TEX root = ../main.tex

\section{Correctness and type preservation}
\label{sec:5.1}
I proof that dependently typed defunctionalization is type-preserving and correct in this section. As shown in Example 4.3.3, it is easy to verify those properties for a particular example program, but they are difficult to proof in general. Firstly, I give the definition of the correctness of the transformation.

\begin{definition}
Dependently typed defunctionalization is correct if for all ground types $A$ and values $v$ of type $A$,
\begin{equation*}
	\cdot \vdash e \goodcolon A \ \land \
	\cdot \vdash e\ \triangleright^* v\ \Longrightarrow\ 
	\sfpl \target{\fdef_{\Gamma}} \cup \target{\fdef_{e}} \sfpr \semicolon \target{\cdot} \vdash \target{e}\ \triangleright^* \target{v'} \text{ where } \target{v'} \equiv \target{v}.
\end{equation*}
\end{definition}
In other words, if a closed program $e$ evaluates to a ground value $v$, then $\target{e}$ evaluates to a ground value $\target{v'}$ that is equivalent to $\target{v}$. Ideally, this follows as a corollary of \textit{preservation of reduction sequences}.
\begin{definition}
Dependently typed defunctionalization preserves reduction sequences if
\begin{equation*}
	\Gamma \vdash e_1\ \triangleright^* e_2\ \Longrightarrow\ 
	\sfpl \target{\fdef_{\Gamma}} \cup \target{\fdef_{e_1}} \cup \target{\fdef_{e_2}} \sfpr \semicolon \target{\Gamma} \vdash
	\target{e_1}\ \triangleright^* \target{e} \text{ where } \target{e} \equiv \target{e_2}.
\end{equation*}
\end{definition}

Ideally, I could show this property by showing that the transformation preserves all small-step reductions $\Gamma \vdash e_1\ \triangleright e_2$, and then proof by induction on the number of reduction steps in a sequence. However, CC's meta-language substitution $(\lam{x}{A}{e_1})\sub{e_2}{y}$ creates a new function definition when it substitutes an expression into a free variable of the function.
So, for a reduction sequence $e_1\ \triangleright \cdots \ \triangleright e_n$ in CC, some $e_i$ may contain function definitions that do not exist in $e_1$ or $e_n$. This means that not all $e_i$ translate into well-typed DCC terms $\target{e_i}$ in $\sfpl \target{\fdef_{\Gamma}} \cup \target{\fdef_{e_1}} \cup \target{\fdef_{e_n}} \sfpr \semicolon \target{\Gamma}$, which makes proofs by induction difficult. Moreover, preservation of reduction sequences is also a key lemma for showing type preservation, since CC's typing rules involve equivalence and the equivalence rule (Eq-reduce) is defined with reductions.

Luckily, meta-theoretic substitution is the only source of creating new function definitions in CC's reduction sequences. This problem would not exist if the source language does not evaluate substitutions into functions but keeps them as primitive expressions instead. To use this observation formally, I define a helper language {\ccs}, which is an extension of CC with the following features:
\begin{itemize}
	\item Substitution is a part of its syntax instead of a meta-theoretic notion.
	\item It does not evaluate substitutions of expressions into functions.
\end{itemize}

Since {\ccs} extends CC, every CC expression is trivially a {\ccs} expression. I denote the trivial transformation from CC to {\ccs} as $\sigma$, then I define the defunctionalization transformation from {\ccs} to DCC. Next, I show that these two transformations preserve reduction sequences and they commute with the transformation from CC into DCC. As a corollary, defunctionalization from CC to DCC preserves reduction sequences.

{\ccs} is an extension of CC with new syntax, type derivation rules, reduction rules, and equivalence rules (Figure~\ref{fig: ccs}). I write {\ccs} expressions in a $\helper{teal, mathematical}$ $\helper{font}$ to avoid ambiguity. {\ccs} extends the CC syntax with \textit{syntactic substitutions} of the form $\helper{e_1\sub{e_2}{x}}$ -- note that this is not the usual meta-theoretic substitution, and I emphasised this by colouring the square brackets teal. 

% Type rules
Type rules for variables, universes, $\Pi$-types, functions, and equivalence in {\ccs} are the same as the standard rules in CC, except that the type of an application $\helper{e_1} \ \helper{e_2}$ is $\helper{B\sub{e2}{x}}$ with the syntactic substitution, since there is no meta-language substitution in {\ccs}.
The type of a substitution $\helper{e_1\sub{e_2}{x}}$ is the type of $\helper{e_1}$ with $\helper{x}$ substituted by $\helper{e_2}$ (\helper{Subst}). 

% Reductions and equivalence
{\ccs} has five more reduction rules about substitutions, which are the standard meta-theoretic substitution rules for variables, universes, $\Pi$-types, and applications being hard-coded into the language. Note that the substitution in the beta-reduction rule $(\helper{\lambda x} \goodcolon \helper{A} . \helper{e_1})\ \helper{e_2}\ \triangleright\ \helper{e_1\sub{e_2}{x}}$ is also the syntactic one. {\ccs} does not reduce substitutions into functions.

{\ccs} has the standard equivalence rules (\helper{Eq-eq}), (\helper{Eq-Eta1}), and (\helper{Eq-Eta2}), defined in the same way as those rules in are CC. Apart from that, it has two new symmetric rules (\helper{Eq-SubEta1}) and (\helper{Eq-SubEta2}) for determining when a substitution into a function 
$(\helper{\lambda x} \goodcolon \helper{A} . \helper{e})\helper{\sub{e'}{y}}$ 
is equivalent to another expression. This is essentially a variant of the $\eta$-equivalence rules that is compatible with substitutions -- $(\helper{\lambda x} \goodcolon \helper{A} . \helper{e})\helper{\sub{e'}{y}}$ is equivalent to $\helper{e_2}$ if applying $\helper{e_2}$ to $\helper{x}$ is equivalent to the function body $\helper{e}$ with $\helper{y}$ being substituted for $\helper{e'}$.

\begin{figure}
\renewcommand{\arraystretch}{1.3}
	\begin{equation*}
		Expressions \quad ::= \quad \cdots\ |\ \helper{e_1\sub{e_2}{x}} 
	\end{equation*}
	\begin{equation}
		\tag{\helper{Apply}}
		\frac
			{\helper{\Gamma} \vdash \helper{e_1} \goodcolon \helper{\Pi x} \goodcolon \helper{A}. \helper{B} \qquad 
			\helper{\Gamma} \vdash \helper{e_2} \goodcolon \helper{A}}
			{\helper{\Gamma} 	\vdash \helper{e_1} \ \helper{e_2} \goodcolon \helper{B\sub{e2}{x}}}
	\end{equation}
	\begin{equation}
		\tag{\helper{Subst}}
		\frac
			{\helper{\Gamma}, \helper{x} \goodcolon \helper{A} \vdash \helper{e_1} \goodcolon \helper{B} \qquad 
			 \helper{\Gamma} \vdash \helper{e_2} \goodcolon \helper{A}}
			{\helper{\Gamma} \vdash \helper{e_1\sub{e_2}{x}} \goodcolon \helper{B\sub{e2}{x}}}
	\end{equation}
	\begin{align*}
		(\helper{\lambda x} \goodcolon \helper{A} . \helper{e_1})\ \helper{e_2}\ &\triangleright\ \helper{e_1\sub{e_2}{x}}\\
		\helper{x\sub{e}{y}}\ &\triangleright\ \helper{x}\\
		\helper{x\sub{e}{x}}\ &\triangleright\ \helper{e}\\
		\helper{U_i\sub{e}{x}}\ &\triangleright\ \helper{U_i}\\
		(\helper{e_1\ e_2})\helper{\sub{e}{x}}\ &\triangleright\ (\helper{e_1\sub{e}{x}})\ (\helper{e_2\sub{e}{x}}) \\
		(\helper{\Pi x} \goodcolon \helper{A}. \helper{B})\helper{\sub{e}{y}}\ &\triangleright\ 
		\helper{\Pi x} \goodcolon \helper{A\sub{e}{y}}. \helper{B\sub{e}{y}}
	\end{align*}
	\begin{equation}
		\tag{\helper{Eq-SubEta1}}
		\qquad\qquad\qquad
		\frac
			{\begin{array}{l @{\qquad} l}
			\helper{e_1} \triangleright^* (\helper{\lambda x} \goodcolon \helper{A} . \helper{e})\helper{\sub{e'}{y}} &
			 \helper{e_2} \triangleright^* \helper{e_2'}\\
			 \helper{\Gamma}, \helper{x} \goodcolon \helper{A} \vdash \helper{e\sub{e'}{y}} \equiv \helper{e_2'}\ \helper{x}
			 \end{array}
			 }
			{\helper{\Gamma} \vdash \helper{e_1} \equiv \helper{e_2}}
	\end{equation}
	\begin{equation}
		\tag{\helper{Eq-SubEta2}}
		\qquad\qquad\qquad
		\frac
			{\begin{array}{l @{\qquad} l}
			 \helper{e_2} \triangleright^* (\helper{\lambda x} \goodcolon \helper{A} . \helper{e})\helper{\sub{e'}{y}} &
			 \helper{e_1} \triangleright^* \helper{e_1'} \\
			 \helper{\Gamma}, \helper{x} \goodcolon \helper{A} \vdash \helper{e_1'}\ \helper{x} \equiv \helper{e\sub{e'}{y}}
			 \end{array}
			 }
			{\helper{\Gamma} \vdash \helper{e_1} \equiv \helper{e_2}}
	\end{equation}
\label{fig: ccs}
\caption{New syntax and rules in {\ccs}}
\end{figure}

Now, I define the defunctionalization transformation from {\ccs} to DCC, which is the transformation from CC to DCC extended with the following two rules. I use $\hbracket{-}$ and $\hbracketd{-}$ to stand for the term transformation and the metafunction for extracting function definitions, and I apply the convention of tagging lambdas with unique identifiers $i\ (i \in \mathbb{N})$ as usual.
\begin{equation}
	\tag{\helper{T-Subst}}
	\frac
		{\helper{\Gamma}, \helper{x} \goodcolon \helper{A} \vdash \helper{e_1} \goodcolon \helper{B} \ \helper{\leadsto}\ 
		 \target{e_1} \qquad
		 \helper{\Gamma} \vdash \helper{e_2} \goodcolon \helper{A}\ \helper{\leadsto}\ \target{e_2}
		}
		{
		 \helper{\Gamma} \vdash \helper{e_1\sub{e_2}{x}} \goodcolon \helper{B\sub{e2}{x}}\ \helper{\leadsto}\
		 \target{e_1}\targetsub{e_2}{x}
		}
\end{equation}
\begin{equation}
	\tag{\helper{D-Subst}}
	\frac
		{\helper{\Gamma}, \helper{x} \goodcolon \helper{A} \vdash \helper{e_1} \goodcolon \helper{B} \ \helper{\leadsto_d}\ 
		 \target{\fdef_1} \qquad
		 \helper{\Gamma} \vdash \helper{e_2} \goodcolon \helper{A}\ \helper{\leadsto_d}\ \target{\fdef_2}
		}
		{
		 \helper{\Gamma} \vdash \helper{e_1\sub{e_2}{x}} \goodcolon \helper{B\sub{e2}{x}}\ \helper{\leadsto_d}\
		 \target{\fdef_{1}} \cup \target{\fdef_2}
		}
\end{equation}

The transformation turns a syntactic substitution in {\ccs} into a meta-theoretic substitution in DCC (\helper{T-Subst}); the function definitions in a substitution $\helper{e_1\sub{e_2}{x}}$ are the union of the definitions in $\helper{e_1}$ and $\helper{e_2}$ (\helper{D-Subst}). 
Since substitutions into functions do not reduce in {\ccs}, the transformation from it into DCC have the following strong properties by definition, which are not true for the transformation from CC into DCC.
\begin{align}
	\helper{\Gamma} \vdash \helper{e_1}\ \triangleright^* \helper{e_2}\ \Longrightarrow\ \helper{\bbracket{e_2}_d} \subseteq\ \helper{\bbracket{\Gamma}_d} \cup \helper{\bbracket{e_1}_d}
	\label{reduction def}\\
	\hbracket{\helper{e_1\sub{e_2}{x}}} = \hbracket{\helper{e_1}}\sub{\hbracket{\helper{e_2}}}{\target{x}}.
	\label{ccs substitution}
\end{align}

Next, I show that the transformation preserves small step reductions in {\ccs} -- if a {\ccs} program $\helper{e_1}$ reduces to $\helper{e_2}$ in one step, then the translated program $\target{e_1}$ evaluates to $\target{e_2}$ in a sequence.
\begin{lemma}[Preservation of small step reductions] If $\helper{\Gamma} \vdash \helper{e_1}\ \triangleright \helper{e_2}$, then
$\helper{\bbracket{\Gamma}_d} \cup \helper{\bbracket{e_1}_d} \semicolon \target{\Gamma} \vdash \target{e_1}\ \triangleright^* \target{e_2}$.
\paragraph{Proof.} This can by proved by an induction on the reduction rules of {\ccs}. All cases are trivial except for the $\beta$-reduction rule 
$(\helper{\lambda^i x} \goodcolon \helper{A} . \helper{e_1})\ \helper{e_2}\ \triangleright\ \helper{e_1\sub{e_2}{x}}$.
In this case, my goal is to show that 
$\target{\fdef} \semicolon \target{\Gamma} \vdash
\target{Apply}\ \targetlab{\flabel_i}{\bar{x}}\ \target{e_2}\ \triangleright^* \ \target{e_1}\targetsub{e_2}{x}$, where
$\target{\fdef} = \helper{\bbracket{\Gamma}_d} \cup \helper{\bbracket{e_1}_d}$ and $\target{\bar{x}}$ corresponds to the free variables in $\helper{\lambda^i}$. By definition, I also have
$\itemdef{\target{\flabel_i}}{\itemtype{\bar{x}}{\_}}{\_}{\target{e_1}} \in \target{\fdef}$ (ignoring type information).
Therefore, 
$\target{\fdef} \semicolon \target{\Gamma} \vdash
\target{Apply}\ \targetlab{\flabel_i}{\bar{x}}\ \target{e_2}\ \triangleright \ 
\target{e_1}[\target{\bar{x}}\slash\target{\bar{x}}\sfcomma\target{e_2}\slash\target{x}]
= \target{e_1}\targetsub{e_2}{x}
$. \qed
\end{lemma}

The transformation preserves sequences of reductions, and the proof follows from a trivial induction on the number of small steps in the sequence.

\begin{lemma}[Preservation of reduction sequences ({\ccs})] If $\helper{\Gamma} \vdash \helper{e_1}\ \triangleright^* \helper{e_2}$, then
$\helper{\bbracket{\Gamma}_d} \cup \helper{\bbracket{e_1}_d} \semicolon \target{\Gamma} \vdash \target{e_1}\ \triangleright^* \target{e_2}$.
\label{lem:ccs prev sequence}
\end{lemma}


The transformation is also coherent, i.e.~it preserves the equivalence relation in {\ccs}.

\begin{lemma}[Coherence ({\ccs})] If $\helper{\Gamma} \vdash \helper{e_1} \equiv \helper{e_2}$, then 
$\target{\fdef} \semicolon \target{\Gamma} \vdash \target{e_1} \equiv \target{e_2}$, where
$\target{\fdef} = \helper{\bbracket{\Gamma}_d} \cup \helper{\bbracket{e_1}_d} \cup \helper{\bbracket{e_2}_d}$.
\paragraph{Proof.} This is proved by induction on the equivalence rules of {\ccs}, and I only show for case (\helper{Eq-Eta1}). Case (\helper{Eq-SubEta1}) follows from similar (and more tedious) proof steps as case (\helper{Eq-Eta1}), and cases (\helper{Eq-Eta2}) and (\helper{Eq-SubEta2}) are true by symmetry. Case (\helper{Eq-eq}) follows directly from the preservation of reduction sequences.

If $\helper{\Gamma} \vdash \helper{e_1} \equiv \helper{e_2}$ by (\helper{Eq-Eta1}) in {\ccs}, then 
$\helper{e_1} \triangleright^* (\helper{\lambda^i x} \goodcolon \helper{A} . \helper{e})$, 
$\helper{e_2} \triangleright^* \helper{e_2'}$, 
and $\helper{\Gamma}, \helper{x \goodcolon A} \vdash \helper{e} \equiv \helper{e_2'\ x}$. 
By Lemma~\ref{lem:ccs prev sequence}, I have 
$\target{\fdef_1} \semicolon \target{\Gamma} \vdash \target{e_1}\ \triangleright^* \targetlab{\flabel_i}{\bar{x}}$ and
$\target{\fdef_2} \semicolon \target{\Gamma} \vdash \target{e_2}\ \triangleright^* \target{e_2'}$, where 
$\target{\fdef_1} = \helper{\bbracket{\Gamma}_d} \cup \helper{\bbracket{e_1}_d}$,
$\itemdef{\target{\flabel_i}}{\itemtype{\bar{x}}{\_}}{\_}{\target{e}} \in \target{\fdef_1}$ (ignoring type information), and
$\target{\fdef_2} = \helper{\bbracket{\Gamma}_d} \cup \helper{\bbracket{e_2}_d}$. 
By the induction hypothesis, I have
$\target{\fdef_3} \semicolon \target{\Gamma} \sfcomma \itemtype{x}{A} \vdash \target{e} \equiv \targetapp{e_2'}{x}$, where
$\target{\fdef_3} = \hbracketd{\helper{\Gamma}, \helper{x} \goodcolon \helper{A}} \cup \helper{\bbracket{e}_d} \cup \helper{\bbracket{e_2'}_d}$. My goal is to show
$\target{\fdef}\semicolon \target{\Gamma} \vdash \target{e_1} \equiv \target{e_2}$ by using DCC's (Eq-Eta1) rule (shown below).
\begin{equation*}
	\frac
		{\begin{array}{l}
		  \target{\fdef}\semicolon \target{\Gamma} \vdash \target{e_1} \triangleright^* \targetlab{\flabel}{\bar{e}} \qquad
          \target{\fdef}\semicolon \target{\Gamma} \vdash \target{e_2} \triangleright^* \target{e_2'} \\
          \itemdef{\target{\flabel}}{\itemtype{\bar{x}}{\bar{A}}}{\targetpi{x}{A}{B}}{\target{e}} \in \target{\fdef} \\
          \target{\fdef}\semicolon \target{\Gamma}\sfcomma \itemtype{x}{A} \targetsub{\bar{e}}{\bar{x}} \vdash 
          \target{e}\targetsub{\bar{e}}{\bar{x}} \equiv \targetapp{e_2'}{x}
    	\end{array}}
		{\target{\fdef}\semicolon \target{\Gamma} \vdash \target{e_1} \equiv \target{e_2}}
\end{equation*}
I already have all the premises $\target{e_1}\ \triangleright^* \targetlab{\flabel_i}{\bar{x}}$, etc., but they are not judged in the desired label context $\target{\fdef} = \helper{\bbracket{\Gamma}_d} \cup \helper{\bbracket{e_1}_d} \cup \helper{\bbracket{e_2}_d}$. I only need to show that $\target{\fdef_1}$, $\target{\fdef_2}$, and $\target{\fdef_3}$ are subsets of $\target{\fdef}$ and to apply the weakening theorem. $\target{\fdef_1}$ and $\target{\fdef_2}$ are clearly subsets of $\target{\fdef}$. I have
$\target{\fdef_3} = \hbracketd{\helper{\Gamma}} \cup \hbracketd{\helper{A}} \cup \helper{\bbracket{e}_d} \cup \helper{\bbracket{e_2'}_d}$ since $\hbracketd{-}$ acts pointwise on contexts. Also, 
\begin{align*}
	\hbracketd{\helper{A}} \cup \helper{\bbracket{e}_d} &\subseteq \hbracketd{(\helper{\lambda^i x} \goodcolon \helper{A} . \helper{e})} \text{ by def.}\\
	&\subseteq \hbracketd{\helper{\Gamma}} \cup \hbracketd{\helper{e_1}} \text{ by (\eqref{reduction def})},
\end{align*}
and $\hbracketd{\helper{e_2'}} \subseteq \hbracketd{\helper{\Gamma}} \cup \hbracketd{\helper{e_2}}$ by (\eqref{reduction def}). Therefore, $\target{\fdef_3}$ is also a subset of $\target{\fdef}$. \qed
\label{lem:ccs coherence}
\end{lemma}

I use $\sigma$ to denote the trivial transformation from CC to {\ccs}. This trivial transformation commutes with defunctionalization, i.e.
\begin{equation}
\hbracket{\sigma(e)} = \bbracket{e}
\label{eq:commute}
\end{equation}
for all $e$ by definition. In addition, 
\begin{equation}
\hbracket{\sigma(e)}_d \subseteq \bbracket{e}_d
\label{eq:commute_d}
\end{equation}
because new function definitions appear in CC's type derivation trees as results of substitutions, but this does not happen in {\ccs}.  I show that it $\sigma$ also preserves sequences of reductions. As a convention, I write $\helper{e}$ for $\sigma(e)$ when there is no ambiguity.
\begin{lemma}[Preservation of reduction sequences ($\sigma$)] If $\Gamma \vdash e_1\ \triangleright^* e_2$, then
$\helper{\Gamma} \vdash \helper{e_1}\ \triangleright^* \helper{e} \text{ where } \helper{\Gamma} \vdash \helper{e} \equiv \helper{e_2}$.
\paragraph{Proof.} Firstly, if $\Gamma, x \goodcolon A \vdash e_1 \goodcolon B$ and $\Gamma \vdash e_2 \goodcolon A$, then
$\helper{\Gamma} \vdash \sigma(e_1\sub{e_2}{x}) \equiv \helper{e_1\sub{e_2}{x}}$. This can be proved by induction on the syntax structure of $e_1$, and all cases are trivial.

Therefore, I have $(\helper{\lambda x} \goodcolon \helper{A}. \helper{e_1})\ \helper{e_2}\ \triangleright 
\helper{e_1\sub{e_2}{x}} \equiv \sigma(e_1\sub{e_2}{x})$, so $\sigma$ preserves small step reductions. Again, the preservation of reduction sequences follows immediately from this. \qed 
\label{lem:sigma prev sequence}
\end{lemma}

I can finally prove the preservation of reduction sequences for dependently typed defunctionalization (from CC to DCC) using the lemmas above. 

\begin{lemma}[Preservation of reduction sequences]For all $e_1$ and $e_2$, $\Gamma \vdash e_1\ \triangleright^* e_2$ implies that
\begin{align}
\sfpl \target{\fdef_{\Gamma}} \cup \target{\fdef_{e_1}} \cup \target{\fdef_{e_2}} \sfpr \semicolon \target{\Gamma} &\vdash
\target{e_1} \triangleright^* \target{e},\\
\sfpl \target{\fdef_{\Gamma}} \cup \target{\fdef_{e_1}} \cup \target{\fdef_{e_2}} \sfpr \semicolon \target{\Gamma} &\vdash\target{e} \equiv \target{e_2}
\end{align}
, where
$(\target{\fdef_{\Gamma}}, \target{\fdef_{e_1}}, \target{\fdef_{e_2}})$ = 
$(\bbracket{\Gamma}_d, \bbracket{e_1}_d, \bbracket{e_2}_d)$ and 
$(\target{\Gamma}, \target{e_1}, \target{e_2})$ = 
$(\bbracket{\Gamma}, \bbracket{e_1}, \bbracket{e_2})$
.
\paragraph{Proof.} Suppose that $\Gamma \vdash e_1\ \triangleright^* e_2$. Then, I have 
$\helper{\Gamma} \vdash \helper{e_1}\ \triangleright^* \helper{e} \text{ and } \helper{\Gamma} \vdash \helper{e} \equiv \helper{e_2}$ for some $\helper{e}$, since $\sigma$ preserves reduction sequences. By Lemma~\ref{lem:ccs prev sequence} and Lemma~\ref{lem:ccs coherence}, 
\begin{align*}
(\hbracketd{\helper{\Gamma}} \cup \hbracketd{\helper{e_1}}) \semicolon \target{\Gamma} &\vdash
\target{e_1} \triangleright^* \target{e}\\
(\hbracketd{\helper{\Gamma}} \cup \hbracketd{\helper{e_1}} \cup \hbracketd{\helper{e_2}}) \semicolon \target{\Gamma} &\vdash \target{e} \equiv \target{e_2}.
\end{align*}
Finally, since $(\hbracketd{\helper{\Gamma}} \cup \hbracketd{\helper{e_1}})$ and $(\hbracketd{\helper{\Gamma}} \cup \hbracketd{\helper{e_1}} \cup \hbracketd{\helper{e_2}})$ are both subsets of $\sfpl \target{\fdef_{\Gamma}} \cup \target{\fdef_{e_1}} \cup \target{\fdef_{e_2}} \sfpr$, I have (5.3) and (5.4) by weakening. \qed
\label{lem:prev sequence}
\end{lemma}

Since ground types and values do not contain functions, $\bbracket{v}_d = \target{\cdot}$, and the correctness of the transformation is just a special case of Lemma~\ref{lem:prev sequence}.

\begin{coro}(Correctness) For all ground types $A$ and values $v$ of type $A$,
\begin{equation*}
	\cdot \vdash e \goodcolon A \ \land \
	\cdot \vdash e\ \triangleright^* v\ \Longrightarrow\ 
	\sfpl \target{\fdef_{\Gamma}} \cup \target{\fdef_{e}} \sfpr \semicolon \target{\cdot} \vdash \target{e}\ \triangleright^* \target{v'} \text{ where } \target{v'} \equiv \target{v}.
\end{equation*}
\end{coro}


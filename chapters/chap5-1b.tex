% !TEX root = ../main.tex

I now present the proof of type-preservation, which requires three lemmas: \textit{substitution}, \textit{preservation of reduction sequences}, and \textit{coherence}. I proved that dependently-typed defunctionalization preserves reduction sequences in Lemma~\ref{lem:prev sequence} with the help of {\ccs}, and now I prove the remaining two lemmas in a similar way. The substitution lemma states that defunctionalization is compatible with substitutions.

\begin{lemma}[Substitution] 
\label{lem:sub}
If $\Gamma, x \goodcolon A \vdash e_1 \goodcolon B$ and $\Gamma \vdash e_2 \goodcolon A$, then $\target{\fdef} \semicolon \target{\Gamma} \vdash \bbracket{e_1 \sub{e_2}{x}} \equiv \target{e_1}\sub{\target{e_2}}{\target{x}}$, where $\target{\fdef} = \target{\fdef_{\Gamma}} \cup \target{\fdef_{e_1}} \cup \target{\fdef_{e_2}} \cup \target{\fdef_{e_1\sub{e_2}{x}}}$.
\begin{proof}
From the proof of Lemma~\ref{lem:sigma prev sequence}, we know that
$\Gamma, x \goodcolon A \vdash e_1 \goodcolon B$ and $\Gamma \vdash e_2 \goodcolon A$ implies that $\helper{\Gamma} \vdash \sigma(e_1\sub{e_2}{x}) \equiv \helper{e_1\hsub{e_2}{x}}$. 
This further implies that $\target{\fdef'} \semicolon \hbracket{\helper{\Gamma}} \vdash \hbracket{\sigma(e_1\sub{e_2}{x})} \equiv \hbracket{\helper{e_1\hsub{e_2}{x}}}$, where 
$\target{\fdef'} = \hbracketd{\helper{\Gamma}} \cup \hbracketd{\helper{e_1}} \cup \hbracketd{\helper{e_2}} \cup \hbracketd{\sigma(e_1\sub{e_2}{x})}$.

By (\ref{eq:commute_d}), $\target{\fdef'} \subseteq \target{\fdef}$. Also, $\hbracket{\sigma(e_1\sub{e_2}{x})} = \bbracket{e_1\sub{e_2}{x}}$, $\hbracket{\helper{\Gamma}} = \target{\Gamma}$, and $\hbracket{\helper{e_1}}\sub{\hbracket{\helper{e_2}}}{\target{x}} = \target{e_1}\targetsub{e_2}{x}$ by \eqref{ccs substitution} and \eqref{eq:commute}. Therefore, we have $\target{\fdef} \semicolon \target{\Gamma} \vdash \bbracket{e_1 \sub{e_2}{x}} \equiv \target{e_1}\sub{\target{e_2}}{\target{x}}$ by weakening.
\end{proof}
\end{lemma}

The coherence lemma states that defunctionalization is compatible with CC's coherence judgements.

\begin{lemma}[Coherence] 
\label{lem:coherence}
If $\Gamma \vdash e_1 \equiv e_2$, then $\targetcon \vdash \target{e_1} \equiv \target{e_2}$, where
$\target{\fdef} = \target{\fdef_{\Gamma}} \cup \target{\fdef_{e_1}} \cup \target{\fdef_{e_2}}$.
\begin{proof}
If $\Gamma \vdash e_1 \equiv e_2$, then we have $\helper{\Gamma} \vdash \helper{e_1} \equiv \helper{e_2}$
by definition, since the source language never contain explicit substitutions. By Lemma~\ref{lem:ccs coherence}, we have $\target{\fdef'} \semicolon \target{\Gamma} \vdash \target{e_1} \equiv \target{e_2}$, where
$\target{\fdef'} = \helper{\bbracket{\Gamma}_d} \cup \helper{\bbracket{e_1}_d} \cup \helper{\bbracket{e_2}_d}$.
Since $\target{\fdef'} \subseteq \target{\fdef}$, we have $\targetcon \vdash \target{e_1} \equiv \target{e_2}$ by weakening.
\end{proof}
\end{lemma}

Finally, I show type preservation with an induction on CC's type derivation rules. 

\begin{theorem}[Type preservation] 
\label{theorem:type preservation}
For all well-typed programs $\Gamma \vdash {e} \goodcolon {A}$,
\begin{equation*}
	{\Gamma} \vdash {e} \goodcolon {A} \Longrightarrow 
	\sfpl \target{\fdef_{\Gamma}} \cup \target{\fdef_{e}} \sfpr \semicolon \target{\Gamma} \vdash \target{e} \goodcolon \target{A},
\end{equation*}
where $(\target{\fdef_{\Gamma}}, \target{\fdef_{e}}) 
= (\bbracket{{\Gamma}}_d, \bbracket{{e}}_d)$ and 
$(\target{\Gamma}, \target{{e}}, \target{{A}}) 
= (\bbracket{{\Gamma}}, \bbracket{{e}}, \bbracket{{A}})$.
\begin{proof}
By proving the following two statements together with a simulteneous induction on mutually-defined judgements $\vdash {\Gamma}$ and ${\Gamma} \vdash {e} \goodcolon {A}$.
\begin{enumerate}
	\item $\vdash {\Gamma} \Longrightarrow\  \vdash \target{\fdef_{\Gamma}} \semicolon \target{\Gamma}$.
	\item ${\Gamma} \vdash {e} \goodcolon {A} \Longrightarrow 
		\sfpl \target{\fdef_{\Gamma}} \cup \target{\fdef_{e}} \sfpr \semicolon \target{\Gamma} \vdash \target{e} \goodcolon \target{A}$.
\end{enumerate}
Statement $1$ follows trivially from the inductive hypothesis. For statement $2$, cases (Var), (Universe), and (Pi) are trivial by induction, and (Equiv) follows directly from the coherence lemma. Cases (Apply) and (Lambda) require some efforts to prove.

Case (Apply). \hspace{0.4cm}
Suppose $\Gamma \vdash e_1\ e_2 \goodcolon B\sub{e_2}{x}$. The goal in this case is to show $\targetcon \vdash \targetapp{e_1}{e_2} \goodcolon \bbracket{B\sub{e_2}{x}}$, where 
$\target{\fdef} = \sfpl \target{\fdef_{\Gamma}} \cup \bbracket{e_1\ e_2}_d \sfpr$. We have
$\sfpl \target{\fdef_{\Gamma}} \cup \target{\fdef_{e_1}} \sfpr \vdash
\target{e_1} \goodcolon \targetpi{x}{A}{B}$ and
$\sfpl \target{\fdef_{\Gamma}} \cup \target{\fdef_{e_2}} \sfpr \vdash
\target{e_2} \goodcolon \target{B}$ by the induction hypothesis.
By (\targettext{Apply}) and weakening,
$\targetcon \vdash \targetapp{e_1}{e_2} \goodcolon \target{B}\targetsub{e_2}{x}$. By the substitution lemma, 
$\target{\fdef'} \semicolon \target{\Gamma} \vdash \target{B}\targetsub{e_2}{x} \equiv \bbracket{B\sub{e_2}{x}}$, where
$\target{\fdef'} = \sfpl \target{\fdef_{\Gamma}} \cup \target{\fdef_{e_2}} \cup \target{\fdef_{B}} \cup \target{\fdef_{B\sub{e_2}{x}}} \sfpr$. We have the goal if $\target{\fdef'} \subseteq \target{\fdef}$. Indeed,
\begin{enumerate}
\item $\target{\fdef_{\Gamma}} \subseteq \target{\fdef}$ by def.
\item $\target{\fdef_{e_2}} \subseteq \target{\fdef}$ since $\target{\fdef_{e_2}} \subseteq \bbracket{e_1\ e_2}_d$ by def.
\item $\target{\fdef_{B\sub{e_2}{x}}} \subseteq \target{\fdef}$ since $\target{\fdef_{B\sub{e_2}{x}}} \subseteq \bbracket{e_1\ e_2}_d$ by def.
\item $\target{\fdef_{B}} \subseteq \target{\fdef}$, because $\target{\fdef_{B}} \subseteq \bbracket{\pitype{x}{A}{B}}$ by def., $\bbracket{\pitype{x}{A}{B}} \subseteq \target{\fdef_{e_1}}$ by $\Gamma \vdash e_1 \goodcolon \pitype{x}{A}{B}$ and Lemma~\ref{lem:subset}, and $\target{\fdef_{e_1}} \subseteq \target{\fdef}$ by def.
\end{enumerate}

Case (Lambda). \hspace{0.4cm}
Suppose $\Gamma \vdash \lami{i}{x}{A}{e} \goodcolon \pitype{x}{A}{B}$.
The goal here is to show that $\targetcon \vdash \targetlab{\flabel_i}{\bar{x}} \goodcolon \targetpi{x}{A}{B}$, where 
$\target{\fdef} = \sfpl \target{\fdef_{\Gamma}} \cup \target{\fdef_e}\sfpr \sfcomma 
\itemdef{\target{\flabel_i}}{\itemtype{\bar{x}}{\bar{A}}}{\targetpi{x}{A}{B}}{\target{e}}$. We have
$\Gamma \vdash \bar{x} \goodcolon \bar{A}$ since $\bar{x}$ are well-typed free variables, therefore, we have 
$\targetcon \vdash \itemtype{\bar{x}}{\bar{A}}$ by the inductive hypothesis and weakening. We also have $\target{\flabel_i}$ in $\target{\fdef}$ by definition.
If $\vdash \targetcon$ is true, the goal can be shown with DCC's type rule (\targettext{Label}). $\vdash \sfpl \target{\fdef_{\Gamma}} \cup \target{\fdef_e}\sfpr \semicolon \target{\Gamma}$ can be obtained from the induction hypothesis and $\Gamma, x \goodcolon A \vdash e \goodcolon B$. Letting 
$\target{\Gamma_{fv}} = \target{\cdot}\sfcomma\itemtype{\bar{x}}{\bar{A}}$, if the following statements are true, then	 we have $\vdash \targetcon$ by (\targettext{WFL-Label}) and weakening.
\begin{enumerate}[label=(\alph*)]
\item $\sfpl \target{\fdef_{\Gamma}} \cup \target{\fdef_e}\sfpr \semicolon \target{\Gamma_{fv}} \vdash \targetpi{x}{A}{B} \goodcolon \target{U} $.
\item $\sfpl \target{\fdef_{\Gamma}} \cup \target{\fdef_e}\sfpr \semicolon \target{\Gamma_{fv}} \sfcomma \itemtype{x}{A} \vdash \target{e} \goodcolon \target{B} $.
\end{enumerate}
By Lemma~\ref{lem:fv}, $\text{FV}(\lami{i}{x}{A}{e}) \vdash (\lami{i}{x}{A}{e}) \goodcolon \pitype{x}{A}{B}$, so $\text{FV}(\lami{i}{x}{A}{e}) \vdash \pitype{x}{A}{B} \goodcolon U$ and
$\text{FV}(\lami{i}{x}{A}{e}), x \goodcolon A \vdash e \goodcolon B$. Therefore, conditions (a) and (b) are true by the induction hypothesis and weakening.
\end{proof}
\end{theorem}
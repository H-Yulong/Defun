% !TEX root = ../main.tex

This dissertation studied type-preserving defunctionalization for a variant of the Calculus of Constructions. I illustrated that Pottier and Gauthier's type-preserving polymorphic defunctionalization does not extend to dependently-typed languages. Then, I presented abstract defunctionalization as an alternative method. Abstract defunctionalization consists of a target language (the Defunctionalized Calculus of Constructions) and a transformation from the source language to the target language. The target language has a primitive notion of function labels that fits the abstract description of defunctionalization. I proved the transformation type-preserving and correct, and I showed that the target language is type-safe as a programming language and consistent as a logic.

In this chapter, I first cover serval topics and issues in implementing the target language, then I end with a discussion on future work.

\section{Implementation of abstract defunctionalization}

This section briefly introduces my implementation of the Defunctionalized Calculus of Constructions (DCC) and the transformation from CC to DCC in OCaml.
The implementation allowed me to check mechanically for complicated and counter-intuitive examples of dependently-typed defunctionalization, which are difficult and tedious to check by hand. 
% Notable files in the code directory are:
% \begin{itemize}
%     \item \texttt{cc.ml}, the implementation of the source language CC.
%     \item \texttt{dcc.ml}, the implementation of the target language DCC.
%     \item \texttt{transformation.ml}, the defunctionalization transformation from CC to DCC.
% \end{itemize}

\paragraph{Syntax} The code structure is based on Bauer's tutorial of implementing dependent type theory in OCaml~\cite{ImplementDependentType}.
The abstract syntax of DCC is defined with the following inductive type \texttt{expr}.
\begin{lstlisting}
    type expr = 
        | Var of variable | Universe of int
        | Label of label * expr list
        | Apply of expr * expr
        | Pi of variable * expr * expr
        | Unit | UnitType
    type context = {
        def : (label * defItem) list; (* The label context *)
        con : (variable * expr) list; (* The type context  *)
    }
\end{lstlisting}
The data types above are the exact resemblance to DCC's syntax in Figure~\ref{fig:dcc syntax}, and it includes the unit type as a base type. Types \texttt{variable} and \texttt{label} are string-integer pairs -- the string is the name of the variable or label, and the integer is used for distinguishing variables or labels named after the same string. Data type \texttt{defItem} is a record for keeping the data associated with a label -- the list of free variables, etc. (definition omitted for simplicity). I implemented utility functions \texttt{subst}, \texttt{normalize}, and \texttt{equal} to perform substitutions, normalize terms, and determine whether two terms are equivalent. 
\begin{lstlisting}
    subst : (variable * expr) list -> expr -> expr
    normalize : context -> expr -> expr
    equal : context -> expr -> expr -> bool
\end{lstlisting}

\paragraph{Type-checking and transformation}
In DCC, an expression has at most one type (up to equivalence), meaning that given a context and a well-typed expression, the type of the expression can be worked out algorithmically according to the type rules. I implemented function \texttt{infer\_type} for inferring the type of a well-typed term and function \texttt{type\_check} for checking whether a given type judgement is derivable. 
\begin{lstlisting}
    infer_type : context -> expr -> expr
    type_check : context -> expr -> expr -> bool
\end{lstlisting}
Given inputs \texttt{(con, e, t)}, \texttt{type\_check} firstly checks if the context \texttt{con} is well-formed. Then, it computes \texttt{t' = (infer\_type con e)} and checks if \texttt{t'} is equivalent to \texttt{t}. 

The defunctionalization transformation is implemented with function \texttt{transform\_full}. Given a type judgement \texttt{(con, e, t)} in CC (it corresponds to $\Gamma \vdash e \goodcolon A$), \texttt{transform\_full} firstly assigns a unique integer-valued tag to each lambda abstraction in \texttt{(con, e, t)}. Then, it computes the term transformation and extracts function definitions for the labelled CC judgement according to the rules defined in Figure~\ref{fig:dcc transformation} and Figure~\ref{fig:dcc def}. It returns a triple \texttt{(con', e', t')}, which are the transformed context, term, and type. I can use \texttt{(type\_check con' e' t')} to see if the transformation preserves types for a particular input.

\paragraph{Checking well-formedness}
DCC's type rules presented in Figure~\ref{fig:dcc typing} are not suitable for practical implementation, since it checks the well-formedness of the context in every sub-derivation tree of variables, universes, and labels. 
This is very inefficient. Since there is no way to extend the label context in a DCC program, \texttt{type\_check} only needs to check its well-formedness once at the beginning of the type-checking process. Similarly, it only checks the well-formedness of the type context once, and makes sure that whenever it extends the type context with a variable-expression pair, that expression is a valid type in the old context. 

\section{Future work}
\paragraph{Recursive functions}
I plan to adapt abstract defunctionalization to the Calculus of Inductive Constructions (CIC), the extension of CC with inductive families and recursive functions. This involves defining a syntax for inductive families, pattern matching, and recursions in the target language DCC. While recursive functions do not add extra difficulty to the defunctionalization transformation~\cite{DBLP:conf/popl/PottierG04}, adding recursions to DCC is a challenge.
Dependently-typed languages only accept terminating functions, since non-terminating functions make the type system inconsistent. Languages like Agda and Coq have \textit{syntactic conditions} to guarantee termination of recursive functions. 
% For example, Agda only allows recursive calls on a strict sub-expression of the argument. I cannot define a recursive function \texttt{f} of type \texttt{Nat -> Nat} in Agda like
% \begin{lstlisting}
% 	f zero = 1
% 	f (suc n) = (f (suc n)) + 2
% \end{lstlisting}
% since \texttt{suc n} is not a strict sub-expression of itself.
DCC should be extended with syntactic termination conditions for recursive functions, and defunctionalization should transform the termination conditions from the source language to those in the target language. 
 
\paragraph{Embedding DCC into Agda}
It is possible to formalize DCC as an embedding in a proof assistant like Agda. The benefit of doing so is having an \textit{intrinsic} abstract syntax that only contains well-typed terms, and the meta-theoretic properties of DCC can be formally studied in a machine-checked setting. 
However, dealing with variable binding and substitutions in the meta-theory is challenging. A promising tool for formalizing variable binding is the second-order abstract syntax~\cite{DBLP:journals/pacmpl/FioreS22}, but it only applies to simply-typed systems at the moment. Other examples of embedding a dependent theory into another used quotient-inductive types~\cite{DBLP:conf/popl/AltenkirchK16} or shallow embeddings~\cite{DBLP:journals/corr/abs-1907-07562}, but it is not clear how to adapt these methods to model DCC's label context. 

\paragraph{Type-preserving compilers for dependent types} 
Ager et al. showed that closure conversion, continuation-passing style transformation (CPS), and defunctionalization could be combined to derive compilers and virtual machines of the untyped lambda calculus~\cite{ager2003interpreter}.
Dependently-typed closure conversion and the CPS transformation are available in the literature~\cite{DBLP:journals/pacmpl/BowmanCRA18,DBLP:conf/pldi/BowmanA18}. With abstract defunctionalization, the next step is to investigate whether the derivation method by Ager et al. leads to a type-preserving compiler for dependently-typed languages, which makes checking the correctness of separately-compiled and linked programs possible.

